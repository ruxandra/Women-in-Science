%%%%%%%%%%%%%%%%%%%%%%%%%%%%%%%%%%%%%%%%%%%%%%%%%%%%%%%%%%%%%%%%%%%%%%%%%%%%%%%%%%%%%%%%%%%%%%%%%%%%%%%%%%%%%%%%%%%%%%%%%%%%%%%%%%%%%%%%%%%%%%%%%%%%%%%%%%%
% This is just an example/guide for you to refer to when submitting manuscripts to Frontiers, it is not mandatory to use Frontiers .cls files nor frontiers.tex  %
% This will only generate the Manuscript, the final article will be typeset by Frontiers after acceptance.   
%                                              %
%                                                                                                                                                         %
% When submitting your files, remember to upload this *tex file, the pdf generated with it, the *bib file (if bibliography is not within the *tex) and all the figures.
%%%%%%%%%%%%%%%%%%%%%%%%%%%%%%%%%%%%%%%%%%%%%%%%%%%%%%%%%%%%%%%%%%%%%%%%%%%%%%%%%%%%%%%%%%%%%%%%%%%%%%%%%%%%%%%%%%%%%%%%%%%%%%%%%%%%%%%%%%%%%%%%%%%%%%%%%%%

%%% Version 3.4 Generated 2018/06/15 %%%
%%% You will need to have the following packages installed: datetime, fmtcount, etoolbox, fcprefix, which are normally inlcuded in WinEdt. %%%
%%% In http://www.ctan.org/ you can find the packages and how to install them, if necessary. %%%
%%%  NB logo1.jpg is required in the path in order to correctly compile front page header %%%

\documentclass[utf8]{frontiersSCNS} % for Science, Engineering and Humanities and Social Sciences articles
%\documentclass[utf8]{frontiersHLTH} % for Health articles
%\documentclass[utf8]{frontiersFPHY} % for Physics and Applied Mathematics and Statistics articles

%\setcitestyle{square} % for Physics and Applied Mathematics and Statistics articles
\usepackage{url,hyperref,lineno,microtype,subcaption}
\usepackage[onehalfspacing]{setspace}

\linenumbers


% Leave a blank line between paragraphs instead of using \\


\def\keyFont{\fontsize{8}{11}\helveticabold }
\def\firstAuthorLast{Bondarescu {et~al.}} %use et al only if is more than 1 author
\def\Authors{Ruxandra Bondarescu\,$^{1,*}$, Jayashree Balakrishna\,$^{2}$, Christine Corbett Moran\,$^{3}$ and Anuja DeSilva\,$^{4}$}
% Affiliations should be keyed to the author's name with superscript numbers and be listed as follows: Laboratory, Institute, Department, Organization, City, State abbreviation (USA, Canada, Australia), and Country (without detailed address information such as city zip codes or street names).
% If one of the authors has a change of address, list the new address below the correspondence details using a superscript symbol and use the same symbol to indicate the author in the author list.
\def\Address{$^{1}$Institute of Cosmology and Gravitation, University of Portsmouth, Portsmouth, PO1 3FX, UK \\
$^{2}$ Harris Stowe State University, St. Louis, MO 63103, USA \\
$^{3}$ NASA Jet Propulsion Laboratory, Caltech, Pasadena, CA 91011, USA \\
$^{4}$IBM Semiconductor Technology Research, 257, Fuller Road, Albany, NY 12203, USA
}
% The Corresponding Author should be marked with an asterisk
% Provide the exact contact address (this time including street name and city zip code) and email of the corresponding author
\def\corrAuthor{Ruxandra Bondarescu}

\def\corrEmail{ruxandra.bondarescu@port.ac.uk, ruxandrab7@gmail.com}




\begin{document}
\onecolumn
\firstpage{1}

\title[Surpassing Subtle and Overt Biases as Women in Science]{Surpassing Subtle and Overt Biases as Women in Science: Lessons Learned}

\author[\firstAuthorLast ]{\Authors} %This field will be automatically populated
\address{} %This field will be automatically populated
\correspondance{} %This field will be automatically populated

\extraAuth{}% If there are more than 1 corresponding author, comment this line and uncomment the next one.
%\extraAuth{corresponding Author2 \\ Laboratory X2, Institute X2, Department X2, Organization X2, Street X2, City X2 , State XX2 (only USA, Canada and Australia), Zip Code2, X2 Country X2, email2@uni2.edu}


\maketitle


\begin{abstract}

%%% Leave the Abstract empty if your article does not require one, please see the Summary Table for full details.
\section{}
The paper discusses factors that keep women and minorities from entering Science, Technology, Engineering and Mathematics (STEM) and considers the climate that induces them to leave research position for administrative ones.  We start by reviewing the experience of the authors and conclude that we barely made it while many equally qualified colleagues quit along the way. Our experiences are similar to anecdotal evidence from talented women across the world. We then consider the climate that induces the loss of women at various career stages. The gender gap is larger in the US than India and Eastern Europe because in developing nations children of both genders are exposed to STEM in their formative years with the attitude that STEM is important, but underlying problems are similar and affect creativity and innovation. Intervention programs have been successful in bringing the ratio of women and men close to parity, but only locally. We discuss the progressive steps taken by MIT's mechanical engineering program where they successfully reached gender parity in the student population and suggest that other universities and departments try to emulate their success. Overall, programs that connect the different levels of education are needed in addition to hiring more women and altering policies to provide women with maternity leave and sick leave starting from when they enter science and engineering as students.  Support networks for existing employees, gender bias training and training that facilitates the learning of new skills for employees with members outside their own subgroup have proven effective in retaining women employees and in improving the environment for both genders.   Women’s schools and colleges play a role in giving women some of the non-cognitive skills for success, and historically black universities and minority colleges offer the same possibilities for other minorities.  The authors of this paper hail from Sri Lanka, Romania, India, and the United States. We hold undergraduate and graduate degrees in physics or chemistry from the United States, India and Switzerland. %We counter the belief that the low numbers occur because women and minorities are not good enough to succeed, and compare the gender gap in the US with data from India, Western and Eastern Europe.
  
\tiny
 \keyFont{ \section{Keywords:} women in science and technology, STEM gender gap, gender parity in STEM, intervention programs, stereotypes, the leaky pipeline, maternal wall, superdiversity, staying in science, attitude towards women} %All article types: you may provide up to 8 keywords; at least 5 are mandatory.
\end{abstract}

\section{Introduction}
The paper focuses on gender bias in STEM. Gender bias is designed to make women in STEM feel uncomfortable by i) keeping women from achieving their potential and then insisting that women are not designed for STEM, ii) making women who are not in STEM feel a sense of accomplishment as compared to `nerd’ women in STEM so that they become part of the bias, and iii) down-playing the achievements of women and minorities who succeed in spite of the bias. A `nerd’ woman has a much lower societal value than a `nerd’ male. People believe in the innate STEM capabilities of men, the lack of feminine appeal of women in STEM, and in the underachieving of women in STEM because they do not want women to be perceived as better than men \citep{rudman2012status}.   Defending the gender hierarchy motivates prejudice against talented women whose work remains unrecognized \citep{rudman2012status}. 

Gender bias appears from a very early age. Studies say girls as young as six disassociate their gender from smartness \citep{bian2018messages}. Minority students who are also targeted by negative stereotypes do the same \citep{fine1984racial}. This is disturbing on multiple scales since not only does it keep talent from certain fields, it also makes people not trust themselves and stops them from reaching their potential. They end up not believing in their own strength and creativity, which leads to less innovation and a lack of belief in the smartness of other women/minorities. Movies like ‘Hidden Figures’ highlight the issues of how the achievements of African-American women have been downplayed. Downplaying achievement is to acknowledge the contributions of some women and minorities while emphasizing that those people were exceptions in the group they represent. It is aimed at preventing a group from achieving its full potential by reducing the competition and breaking the spirit, adding roadblocks, making people uncomfortable. The formative education of women is undermined by making them constantly feel wrong about feeling interested in STEM fields. African-Americans and Hispanics have a lower economic base and thus face the added load of racism and economic bias much more than white men and women \citep{BlackEco2,BlackEco1}.

 The gender gap in Science, Technology, Engineering and Mathematics (STEM) with a significant male to female ratio is then attributable to societal cues as to what constitutes being male and what constitutes being female. Early on these cues are provided in families by the gender specific roles of the parents, and by the toys and activities boys and girls are encouraged towards. The shaming of girls when they act like “boys”, media depiction of males and females, indulgence of boys when they are exploratory and when they break things, while letting girls know that such behavior is unacceptable for them leads to building societies where STEM exploration and creativity are considered the man’s purview. Any challenge to the status quo results in a reaction that is detrimental to women. Studies find that by the age of 6 girls dissociate their gender from cleverness \citep{bian2017gender,bian2018messages}. By the age of 10-11, STEM loses the interest of girls who already believe that a science career is not for them \citep{archer2012balancing}. By highschool there is a significant gender gap. While girls outnumber boys when it comes to taking Advanced Placement (AP) tests, they are underrepresented in advanced placement in STEM fields other than biology and environmental science. Boys outnumbered girls by more than 4:1 among AP tests in computer science, and by more than 2.5:1 in physics AP and 1.5:1 in calculus \citep{ericson2013detailed}.

%because the society is not interested girls associate science with cleverness and see  The STEM gender gap widens 

%, insignificant at elementary school level starts raising its ugly head in middle school and widens in high school. The teen years are when STEM loses the interest of girls because they start to understand that boys and the larger society may not be interested in the STEM girls. % to fit in with their peers start to understand the price they must pay for acceptance. %The teen years are when STEM loses the interest of girls because they start to understand that boys and the larger society may not be interested in the STEM girls. 


%Studies show that young children show similar interests in science regardless of gender. Later, a combination of factors that include the attitude of adults, gender-specific toys and clothes that assume boys are the heroes, the lack of women mentors and/or in leadership roles, and the portrayal of women in the media cause a gender gap. This gap widens as the children grow. The gender gap becomes visible in highschool when science classes are already dominated by boys. 

The gender gap widens in college and graduate school when it becomes common for the women who succeed to stay in science to end up the only ones in their class or research group.  The overall fraction of women with STEM degrees flattens at around 35\%.  The gender ratio varies between the subfields with physics, computer science, and engineering retaining a substantially lower number of women than other fields.  The percentage of women who obtain bachelor degrees in physics hovers around 20\% since 2005 \citep{apsData}. The data shows that the gender gap will not decrease unless new methods are employed.

Intervention programs that focus on attracting talented high school students to STEM and inviting women to campus to visit coupled with a modernization of the course-work were shown to lead to gender parity at college level.  The MIT Women's Technology program invites talented high school women to a four week classes and labs summer at the Massachusetts Institute of Technology (MIT). They find that existent women students and faculty attract more women, and that active recruiting of women is needed.  The modernization of the coursework with focus on improved pedagogy increased enrollment for both genders, and their ability to thrive. As a result, MIT has successfully brought the enrollment of women in its mechanical engineering program from 13.2\% to 49.5\% \citep{MITpress}. Harvey Mudd College has also recently achieved gender parity in its student population \citep{loftus2015piercing,helble2017our}.
% The hiring of women results from inviting talented women to apply and can lead to a snow-ball effect where the presence of women shows that they can be successful and more women apply.
%and that allowing for basic human rights like maternity leave is a necessary part of the process.
%Many years ago Mileva Einstein returned to Serbia when she was pregnant with her first child. Albert Einstein did not want to marry her until he had a stable job, and the job applications
%took time. The child was a daughter whose trace was lost; it is believed she died. Various sources also show that Mileva and Albert worked jointly and submitted papers together only with his
%name on them. Yet she hit the maternal wall and did not receive her PhD. Her daughter is believed to have died or been abandoned. Even though studies on maternity leave found that a paid leave of 40 weeks saves the most lives, it is yet to be accepted that maternity leave is not a luxury. It is a basic right for both mother and child and it is also a life-saving necessity.  It is not sufficient to have a work-around that requires women to fight for basic human rights like maternity leave because many will fall through the cracks at a time when the STEM community cannot afford to have Mileva's story repeat so easily. Scientists from Zurich can still end up having to return to their home countries when they are sick or when they have children without an income and without health insurance.

%In Asia and Eastern Europe families strongly encourage their children to study whether they are boys or girls. They study with their children and/or hire private tutors even when their incomes are lower than the national average. n Eastern Europe, up to highschool, the number of boys and girls who study math and science is the same. Most women then ditch science and math when they start highschool. Top highschools that focus on mathematics, physics and computer science have a percentage of women of around 30\%.  This coincides with the period when women are expected to develop skills that make them appealing to men, which makes a science career less desirable due to the misconception that smart women are not attractive. 

%It is unlikely that an understanding of physics, computer science, chemistry or mathematics lowers a person's attractiveness. In fact, people are attracted to those who can help them grow and with whom they can share their experiences. However, stress, which is prevalent for minorities in any field and for people in leadership roles, has been seen to increase health problems and lower attractiveness. 

%Eastern Europe and Asia have a long tradition in promoting education, but have yet to embrace equality for men and women. Some argue that the higher rate of women that work in STEM is a result of communism, which forced both genders to work. Women who witnessed communism argue it endowed them with a double burden of having to work outside the home and taking care of a partner, home and children with little to no support from the community. Communism stifled creativity further through closed borders and selective information flow driven by politics.  

%Eastern Europe and Asia have cultures where families strongly encourage their children to study. In Asia, the pressure on 
%men is higher than on women because they are seen as the supporters of the family. Misplaced expectations from the society cause girls to lose interest in science at an early age.

In time, the solution to gender balance at college level might be to scrap standardized tests, which were shown to keep women and minorities out of the best schools, and replace them with tests that  build on exams like those administered in college or graduate schools on which the GPA is based. Grade Point Average was seen to correlated with graduation rate, while standardized tests were repeatedly shown to fail at predicting success \citep{ripin1996fighting,miller2014test,petersen2018multi}. %Programs that connect the different levels of education by inviting high school/college students on campus for hands-on experiences were also seen to increase admission of women and minorities. % Standardized tests are tools introduced in 1940s to deal with the large number of international students escaping Europe and Asia to come to the US. The tests were repeatedly shown to fail at predicting success, while keeping women and minorities out of the best schools \citep{ripin1996fighting,miller2014test}.

While most women and minorities in the western world choose to stay out of STEM fields, the inclusion of women and minorities is sorely needed \citep{SheNumbers, 2018Report}. It has been shown that collaborations work efficiently only when there is enough diversity to prevent grouping based on prejudices \citep{page2007making}. This ensures social coherence. %To avoid losing our best and most talented individuals mid-career, employers have to encourage a healthy balance between work and family life where paid maternity/paternity leave, egg freezing, sick leave, child and elderly-care are available for all employees independent of nationality, visa status or gender. Today, instead of revising policies to ensure all this, the blame is shifted from the inadequate policies that can and should be changed to the employees. This culture of blame has devastating consequences for women and minorities, and particularly mothers who are already more vulnerable to depression and self-blame \citep{hyde1995maternity}. Instead of finding more reasons for blaming them, companies and universities ought to focus on changing rules to help them stay in their field of expertise.
The final goal of society should be to have as many people as possible choosing what they would like to pursue and to be productive and innovative in their chosen fields. No one should be stymied in their endeavors because they have to cater to ordained repressive societal norms. A change in attitudes will only come when a sizeable number of women make the leap. This requires an understanding of the current conditions and support of innovative intervention strategies.

This paper proceeds as follows. Section \ref{Sec10} describes the experience of the authors. Section \ref{Sec9} moves onto the struggles of famous women like Marie Curie and Mileva Einstein with mixing family and career and then to anecdotal quotes from some of today's prominent women scientists. Section \ref{causes} focuses on the causes of gender bias and of the STEM gender gap. Section \ref{Sec2} describes the attitude that makes women and girls think they cannot fit in STEM from the very beginning. Next, section \ref{Sec3} mentions the tendency to maintain the status quo and the consequences of a restrictive history on women. After that the isolation of women and vulnerability of women and minorities is briefly discussed.  Section \ref{leaky} describes the leaky pipeline problem where more than half of the women in STEM switch careers by their mid-thirties with very few of the ones who remain in their field of expertise reaching the top level of their profession. In Section \ref{Sec4} the gender ratio in the US is compared to that in India and Europe. %The larger number of women in STEM in India and Eastern Europe is then attributed to the economic push towards STEM. 
A discussion of successful intervention programs follows in section \ref{Intervention}, where the first two sub-sections focus on attaining gender equality in high school and college. MIT's success at attaining parity in mechanical engineering is discussed next. Afterwards, section \ref{6.2} focuses on admission and retention in graduate school, and section \ref{6.3} mentions retention beyond graduate school where the next major wave women and minorities is lost. Section \ref{Sec7} considers single sex education and minority colleges. They are the oldest intervention programs and continue to be successful. While some parallels are made between black Americans and women, discussing racism is beyond the purpose of this work. Then section \ref{Sec5} considers the impact of diversity on innovation and shows that the potential of women and minorities remains largely untapped. Studies show that immigrants outperform nationals in both genders, while performance is optimal when a team is not dominated by one gender or one or two national identities. Concluding remarks follow.
%ntervention programs do work and that with programs that connect talented highschoolers to college through internships and summer long courses, 
%Our children's time is valuable. School should not be something to endure that keeps children busy so that their parents can work.
%This portrayal has significant impact on the younger generation who grow up in front of their tablets, phones and TVs .
%Since the competition is larger, the options in modeling are more limited than if women were to become scientist or engineers.

%\section{What Causes the STEM Gender Gap?}
\section{Lessons from talented Women who struggled to succeed}

% Instead of making parents feel guilty that they dared to have children, and can no longer spend all their energy on work, we ought to change the rules and help them fit back. for adequate human rights to all employees including graduate students and postdoctoral scholars, who spend many years under the temporary employee label with preciously few rights. 
% will make it easier for employees to work when having a family and to contribute to mentoring the next generation when their health fails,
 \subsection{The Experience of the Authors}
 \label{Sec10}
 The authors of this paper hail from Sri Lanka, Romania, India, and the United States with formative education in these countries.  All have undergraduate and graduate degrees in physics or chemistry from the United States, Switzerland, and India.   The authors studied, taught and performed research at Cornell University, Massachusetts Institute of Technology, Washington University in St Louis, Harris-Stowe State University, Mount Holyoke College, the University of Illinois at Urbana-Champaign, Pennsylvania State University, the University of Z\"{u}rich, Caltech, the Jet Propulsion Laboratory, the South Pole Telescope, IBM and SpaceX. Since this paper discusses gender balance in STEM fields across cultures, we find it valuable to discuss our background and a few personal experiences below.

{\bf Ruxandra Bondarescu describes highschool in Romania, college at the University of Illinois at Urbana-Champaign (UIUC), graduate school at Cornell University, and mixing postdoctoral studies at Penn State University and the University of Z\"{u}rich (UZH) with motherhood.}

I attended the Grigore Mosil Highschool in Timisoara. It was considered the best highschool in the city for studying STEM. All students had the same curricula and had to study informatics for 8 hours every week, mathematics for 5 hours and physics for four hours a week in slight detriment to general education (e.g., music was not taught; geography, history and biology were reduced from two to one hour a week, etc).  The 30 students in my class happened to be equally divided between sexes: we were 15 young women and 15 young men. Out of these students, three went to college and graduate school in the US and all happened to be female. Today, the class graduating in 2018, has 141 students (which are divided in 5 groups or classes) and is about 45\% female. When I went to Z\"{u}rich as a Dr. Tomalla postdoctoral fellow, the only female faculty in Particle and Astrophysics at UZH was a graduate of Grigore Moisil. 

My bachelor degree is from the University of Illinois at Urbana-Champaign (UIUC). I went to Illinois to start work at the National Center for Supercomputing Applications (NCSA) with a supportive advisor who had worked with my brother in Germany, but retained an affiliation with UIUC. After finding I scored 860 on the GRE Subject Test in Mathematics at the age of 18, he had the courage to invite me to work with him and spent some of his time trying to bend rules so that I would be admitted. I had been admitted to other universities in the US, but none offered a full scholarship and the income of my family was about \$700/month. However, for an international student, UIUC was no better. Tuition was high, and I could not afford it. I entered the US as a student in a nearby community college for which I could afford to pay, and I enrolled at UIUC on a part-time basis for a semester for which my advisor paid the tuition from a Microsoft Grant he had received as an award. At UIUC, I only chose senior-level courses and obtained outstanding grades (all As).

The next semester, I asked the physics, computer science and mathematics departments if they offered tuition waivers. I enrolled as a physics major because only the physics department could award teaching assistantships that came with a tuition waiver. I received one with the condition that I graduate in a year.  In computer science, they said they were a huge department with more than enough students and did not care if I stay or leave, and in math I could only receive an assistantships that paid hourly at minimum wage, which was little help towards tuition. %The assistantship was obtained with support from office-mate Galina Wind, physics department head Gary Gladding, and undergraduate advisor Linda Lorenz.

%Before obtaining a Bachelor degree in physics with a double minor in mathematics and computer science, I was a full time student for one year at UIUC.  I also continued my work at NCSA with Prof. Seidel and Dr. Gregory Daues, who was extremely supportive in all my endeavours. Our work resulted in several publications. 

I graduated in a year with a degree in physics and a double minor in mathematics and computer science. I also continued work at NCSA, which resulted in several awards and publications. At the graduation ceremony, we were 4 women, which constituted about 14\% of the physics students graduating in 2003. Of my class, I was the only woman going onto physics graduate school. My graduation in such a short time was seen as a sort of miracle. It was attributed to hard work and to dedicated staff from admissions and from the physics department who bent rules for me and allowed me to test out of the courses I needed. At UIUC, I did not take classes from or work with any female professors. However, I knew there was one woman in physics and one in astrophysics. Today, among the faculty, the UIUC physics department has 9 women out of 57 professors, and is 15.8\% female. 

I then joined Cornell University's physics department where I was awarded a two year fellowship. In this period, I did not have to teach, and could focus on classes and research. I had been admitted to eight other universities. I went to Cornell because of the fellowship and because the Physics Department Head added a hand-written note to the admission letter inviting me to come. He also took the time to show me around when I visited the department, and explicitly said he wanted me to be part of his group.  I graduated five years later under the supervision of the Head of Physics and the Head of Astronomy, who made a superb team being both talented and kind.

I was one of the 6 women entering physics graduate school that year out of 40 students. I support the multiple advisor system. My two advisors made a superb team that helped me stay in science and excel. From the women colleagues in my year: four graduated with a PhD, one took personal leave and never returned, and one was a transfer student who obtained a Masters from Cornell and went on to finish her PhD at her home institution. Three of those four women who succeeded to obtain a PhD in physics from Cornell went on to pursue STEM careers, and two are still working in STEM today. 

Cornell's physics and astronomy departments are known across the US as non-toxic and particularly supportive of their students. Over a 20 year period, the graduation rate is 73\% for the physics students who are women, which is higher than national graduation rate of 59\% for PhDs over all fields. As part of the graduate women in physics group, we received funding from the graduate school, and from the physics department to invite one woman speaker of our choice per semester for the main physics colloquium and to meet with existent female faculty and speakers for lunch or dinner. We were given the opportunity to propose and invite leaders in the field for seminars. This was particularly empowering for graduate students. We also had a panel meeting per semester where Cornell's few women faculty would give us advice on how to succeed.  Beyond this, all Cornell graduate students can apply and receive funding from the graduate school to partially fund travel to conferences, which, combined with funds for students from the American Physical Society, made my presence at all major meetings in my field possible. 

Today, the physics department at Cornell is 17\% female boasting 8 women professors out of the 45 tenured and tenure-track faculty. There were six female faculty when I graduated across physics, applied physics and astronomy. The number of women professors was low enough that in the five years I was there I have never taken a science class taught by a woman. I have had no women professors at either Cornell or UIUC. 

After completing my PhD, I started a postdoctoral position at Penn State University. Two years later my first child was born. My postdoctoral advisor was extremely supportive. He worked with staff to find ways to pay me for the next year and provide the maximum possible leave. The H-1 B visa requirements were of such a nature that he had to explicitly write how unqualified I was to be able to keep me on because postdoctoral scholars are at the bottom of the salary bracket. I had a month of unpaid leave, and was encouraged to come to work only for meetings and to bring my child to work. My salary was also increased.

I had no access to day-care because the one on campus had a long waiting list. So, I ended up enlisting the help of extended family members to care for my son, which was a challenge. I also had a private office where I could breast-feed if needed. At Penn State, we were encouraged to invite speakers for the seminar of the center there, go to lunch and dinner with speakers, and funding was available to support one or two yearly visits from collaborators. This helped me write articles with colleagues at Caltech, the University of Mississippi, and Syracuse University. As the only woman scientist in the institute of the time, I had the opportunity to be at dinner with primary donors for the university, and I also led the team of postdoctoral scholars and students in presenting our research to representatives of the National Science Foundation.

I was next awarded a postdoctoral fellowship at the University of Z\"{u}rich (UZH). I stayed at UZH for five years on a series of successive one and two year contracts. When I arrived with my one year old son, I brought him to work for the first few days. My office was shared and it was disturbing to my colleague, who stopped coming to work. So, I was told I needed to do `something' with my child.  The small day-care on campus was very friendly, but had a  waiting time of more than one year after registering with priority given to permanent staff (postdoctoral scholars are temporary employees). So, my mother retired from her job as a doctor, and came to help care for my son. The environment was supportive. I was allowed to invite one or two collaborators per year for the Particles and Astrophysics Seminar at UZH, and in my last year, I received partial funding to give invited seminars at universities across the US. My research at UZH was featured several times in various journals across the world, which included the IEEE Spectrum, the R\&D Magazine, New Scientist, and the MIT Technology Review. The professor I worked with was praised by the university for our joint work, and, while I was there, he was promoted to the US equivalent of associate professor after 20 years of being an assistant professor, which is tenured in Switzerland, but comes with lower pay. 

When my UZH position ended, I was eight month pregnant with my second child. I gave birth about 6 weeks after my appointment ended, and as temporary staff I had no right to maternity leave since my appointment would have ended at that time even if I had not been pregnant. There was no possibility to obtain another position while pregnant. So, I returned to my home country without health insurance and unemployed, I explained to my mother that I had been temporary staff even though I had been employed for five years in the same place. My mother saw this as a violation of not only my rights, but also of those of my child, which should not happen, and was outraged as a doctor, a woman and a mother by the rules that are commonly applied in STEM.  So, eventually, I gathered enough courage to contact Human Resources (HR) after encouragement from a program for fixing the leaking pipeline who had no funding to help. They wrote back that indeed if the delivery is 6 weeks or more after the contract ends, the university has no obligation towards temporary employees. This is stated in a footnote in the university rules regarding maternity. However, they said that if I had I reported the pregnancy early enough they might have been able to find a work around. They emphasized that departments have {\it no} requirement to report the pregnancy of their employees to HR with most pregnancies being unreported. They also said that my paid seminars to the US meant I left work even earlier and did not move when I was eight month pregnant as I should have according to the contract and that I was in the wrong because of this. I mention that as a scientist, it is expected to travel and give seminars, and that I am guilty of clustering them at the end of the contract so that I could move earlier, and did not have to lift furniture too close to the end of my pregnancy. 

I was then advised to contact the Swiss National Science Foundation. The Head of the Equality Office personally responded to my questions and said there was no funding she could access for such situations, and that she quit science because she has two children herself, but was optimistic about the future. Overall, the university staff clearly explained that it was my fault that nothing could be done for not reporting the pregnancy early enough and for not investigating a potential work-around. I was also told I could have legally staid unpaid in Switzerland for three months after my contract ended, and since I chose to not avail myself of this opportunity, it was my fault for potentially being without health insurance elsewhere. I still stand by my choice since I do not think I would have found a job with a newborn child in those three months and to return home with no savings, a baby and young child would not have been a better solution.%I could have legally stayed unpaid in Switzerland for another three months, continue to pay for health insurance and for the living expenses from my savings, give birth there, and then returned to my home country with a baby and a young child. 

I was the second woman to give birth from my group. The other colleague came from Pakistan for a one-year exchange program. She hid her pregnancy until she arrived because she would not have received the position otherwise, and when she had the baby no extension or maternity leave was granted.  She did ask our advisor and the program officers if she could get an extension, but when the answer was `no' she never followed up with HR. So, like mine her pregnancy was never recorded. She was not able to come to work often because she had a high risk pregnancy, and her previous child died a few days after birth. She had also suffered a number of miscarriages before that. Our advisor allowed her to skip work whenever necessary, and even said that if all she got form the program is a healthy baby that is fine with him. 

She was placed in my office because it was assumed that as the only other woman in the group I could advise. I still remember her crying after searching for an apartment to live in by foot, which in Zurich is quite challenging.  She was in the last month of her pregnancy at the time. Eventually, our advisor found her a one-bedroom apartment. When she looked ill, I sent her to the hospital, and emphasized the importance of prenatal care over work. Once she delivered her baby there was no available day-care, but her husband was eventually allowed to come from Pakistan to help. He was not used to performing childcare duties and so she had to stay home to help him most of the time. She did finish a technical article based on work performed at UZH after returning to Pakistan. She told us the environment was less toxic in Switzerland than in Pakistan where her female colleague was not allowed to skip work to be with her very young child, and often came to work crying, but that in Pakistan there was help from extended family and so she could perform some work.

{\bf Prof. Jayashree Balakrishna discusses her education in India, graduate school in the US, and teaching experience at Harris Stowe State University.}
My education started in India. My father was an engineer and my mother stayed home to raise my two siblings and me on a single income. When I graduated high school my field was STEM-biology (Math, Physics, Chemistry, Biology, English).  In my class there were about 24 boys and 16 girls. There was a definite bias from faculty and students that males were better at math and engineering that they had no qualms displaying. People were comfortable with women wanting to be doctors, but not engineers. This bias was also exhibited by some of the women. In the STEM-engineering class (Math, Chemistry, Physics, Engineering-Drawing, English) there was only one woman and the rest were men. She ended up studying English in college. There was an option of choosing domestic-science instead of Mathematics even in the STEM field. Two women took this option from my class and one of them ended up becoming a doctor. Two male student in the next batch took that option and created a hue and cry in the school with even some of the women saying they did not know what they were doing.  

However, at home the expectation was that STEM is important and that one should do STEM. Nobody at home could change prevalent attitudes and so they would not empathize with the situation. You were supposed to play the best hand you could with the cards you were dealt. It always seemed that guys in class could talk to each other and bounce off ideas but the few women aspiring for Math-Engineering glory were isolated.  Since it was a preposterous idea for a woman to think she was able to excel in these fields, this meant the women in STEM tried to avoid the public eye. Some even seemed to underperform for fear of offending the norm. I attended a women's college for my undergraduate degree. One thing to note is that hostel accommodation for undergraduate women was available in women's colleges with a few seats in a postgraduate women's hostel. The co-ed colleges that had hostels had men's only hostels. Thus far more undergraduate male students stayed in on-campus accommodations than women. In the graduate master's program in India there was a higher relative percentage of women in theoretical physics streams like particle physics than in applied fields like electronics. I graduated from high school and college in India in the 80s. These incidents highlight attitudes from when I went to highschool and college in India. The job market has since grown and multinational companies and growing Indian companies in the technology sector have created many more STEM related jobs. Since then the biases may have become more subtle, but the attitude in the home is similar in that it is still assumed that STEM fields are the important ones to study. 

In the US I attended Washington University in St Louis for graduate studies. The enrollment of women in the physics graduate program was low with no women faculty outside the earth and planetary-science department. Now (from going through the listings on their website and the ratio of male:female graduate students is about 3:1). As of August 2018, there were 33 current faculty listed out of whom there is 1 female Research Assistant Professor from Earth and Planetary Science, 1 female Research Associate Professor from earth and Planetary Science, 1 female Research Professor from Earth and Planetary Science, and 1 female Senior Lecturer. The rest were all males. However, the situation appears to be improving this year with three new female hires.

While finishing my research at Washington University, I taught as a full time lab instructor (Sabbatical replacement position) at Saint Louis University for 1 semester (I had 5 male and 4 female teaching assistants - mostly undergraduates). My Electricity and Magnetism, Mechanics, and Physics for pilots labs, had a very high male to female ratio. After graduate school I obtained a tenure-track faculty position at an HBCU college where I was an adjunct instructor when I was writing my thesis. I have gone through the stages of assistant, associate, and I am now a full professor. Our highest enrollment is of minority women.  In HBCUs the ratio of black women: black men is close to parity \citep{simms2014educational}.  

Many of our students come from underserved communities, and likely went to public schools with a high number of minorities. This community is underserved from elementary school onward and so the issues are different from being a woman in STEM.  African-Americans have been kept from higher education longer than women in the US. Women were first admitted into Washington University in Saint Louis' medical school in 1918, while it was 1947 before the first African-Americans were allowed in the medical school. Washington university was fully desegregated only in 1954. There is a need for more science and math teachers in public schools and for after school programs that focus on math and science, and work in collaboration with colleges. This would be a good intervention program and should include competitions to encourage the students who have STEM aptitude that are similar to the athletics programs.

%https://www.statista.com/chart/4467/female-employees-at-tech-companies/

{\bf Christine Moran describes high school in Columbus, Ohio, USA, college at the MIT, graduate school at the University of Z\"{u}rich (UZH), postdoc work at Caltech, and work in industry.}
In my experience in high school, I wanted to fit in, so I often played up my absent minded professor side to the point of being thought a "ditz" (a term primarily applied to women). Math and science always came easy to me, but I would hide the fact I often had the highest grade in the class. I had a small group of personal friends, most of whom were highly intelligent artists and activists, and none of whom were in my classes. I was in the robotics club, and had a part time job when I was 16 working at an engineering firm, but I thought math and science were boring because they were so easy and I wanted study philosophy then go to law school like my dad. 

I knew I could go to college for free if I went to an elite institution (Harvard, Stanford, MIT etc.) because of my single mother's finances and the financial aid these institutions provide. I read up on the typical profile of the student who was admitted to these colleges, and tried to fit it. Luckily I did well on standardized tests and although I found school boring, with a goal in mind it was easy to focus. I went to the information sessions for the elite colleges, and found most of them off putting in their elitism (I am not sure what I expected), except for MIT which I found down to earth, wacky and fun. The MIT spirit reminded me a lot of my friends and I thought it would be "funny" for a person who wanted to study philosophy and was known as a ditz to go to MIT. I applied early action, and was accepted. I often would get people reacting in surprise at the fact I would be attending MIT. One former physics teacher told me ``I hear they are letting in artsy people now". As a teenager I enjoyed surprising people or catching people off guard with my intelligence; as I became an adult this ``comedy" routine wore me down, because the humor came from people underestimating me, which was usually because of my gender, appearance, and affect. 

At MIT I found a place I felt home. Everyone took the same common mathematics and science core, which quickly lead me to realize these subjects could be challenging and rewarding and that I wanted to study more.  My admitted class was approximately 50\% women, and although my two majors (Physics, Computer Science and Engineering) were a smaller percentage, it was never small enough that I felt out of place. I think had I gone to any other college, I would not have entered the sciences and I'm thankful that my high school sense of humor brought me to MIT. At MIT I had many summer internships and chances to work in teams. In one summer internship, I traveled to a software engineering company in Norway where I was the first woman every hired in any department by the office in which I worked. This didn't feel off, although I recognize in retrospect I did a lot to seem ``one of the guys". After graduation, I worked at a local company called BBN Technologies doing machine translation research. I made an excellent salary and saved a lot of money. My team had several senior women on it, and was diverse in terms of age and cultural background. I ultimately decided to go to graduate school, and moved on to pursue my Master's and later PhD at the University of Z\"urich (UZH).

It is not usual to receive funding as a Master's student in Europe where I had set my mind on going, but I was offered funding for 1 year (2/3 of my time) as a Master's student at the onset with the understanding the final 6 months of funding would follow at UZH. This clinched my decision to begin at the UZH. At UZH there was just one other woman studying for a Master's in my field at the same time. She and I bonded and did much of our studying together. I picked my Master's thesis with the understanding I would be working with the same supervisor as a Ph.D. During a meeting with this supervisor, it came out that he wanted me to work as the assistant to his secretary (who he happened to be dating) and do menial work for her in exchange for the final 1/3 of my funding, which I found demeaning given my skillset and refused. He offered the same ``opportunity" to the only other woman in my program, who despite my advice, took it. I am convinced this offer was related to my gender. At the same time, despite the fact I was doing well on my Master's thesis (I would receive the highest possible grade for its execution), he made clear that he would not be hiring me as a PhD student. The same professor had also previously fired his only female PhD student ever, and had hired another female PhD physics student to clean his apartment (the rumor was, naked). The woman who took the job as the assistant to the secretary, struggled to be taken seriously in a scientific role because of her administrative duties, although working with her I can testify that she was equally talented as the men and myself in the department.

This sent me in a big scramble for money and into a crisis of grief and wondering what I would do next, I found it difficult to go to work. Another Professor in the department offered me a Ph.D. project, but as he was not the Professor I had set on working with and he had a rumor of being difficult with students and now I felt the environment to be toxic, I was unsure whether I wanted to stay. I began to eat into my USD savings to cover my living expenses, at the time that the USD was very weak, and quickly began consulting to make more money, while I finished my Master's. I went to a nearby University, the ETH, to work for a professor there for a few months while I decided whether I wanted to return. Ultimately I did return and accepted the PhD offer of the second professor. I kept up my consulting projects, and took several unpaid leaves of absence to continue them, as well as to do scientific outreach projects. I graduated with my PhD. Many years later, my consulting work produced more than half a million dollars of revenue for me over my PhD, as I collected equity in the companies for which I worked, which when the companies were successful made me successful. I was happy I didn't take the job serving as a secretary's assistant. The woman who did was refused a PhD position and ultimately dropped out of science.

Myself and my advisors assumed with my consulting work, that I would likely enter industry after graduation and I did not actively seek or receive counsel on postdoctoral fellowships. I did indeed begin to apply to industry, but as I had a budding interest in numerical relativity and had made some stabs at research in that direction and contacted professors abroad, decided to reach out to some of my contacts. One of them offered me a 1 year postdoctoral position at Caltech. I was delighted, and ended up deciding to move to LA to first work at SpaceX and later take the postdoctoral position. I was also excited because this supervisor was an avowed feminist and seemed to have many talented women working in his group. I moved to LA and worked for SpaceX for 3 months in an internship. There were very few women at SpaceX and none on my small team, although there were many among the intern class their numbers were diluted throughout the company. During my time at SpaceX I received active mentoring on submitting proposals and on potential postdoctoral fellowships to apply for by my future supervisor at Caltech, and I aided in submitting proposals and submitted my own. I started at Caltech, and the very next day heard that I had won one of the most prestigious: the NSF Astronomy and Astrophysics postdoctoral fellowship, which would fund me for a full 3 years at Caltech at roughly double my initial salary, with a mandate to do outreach work with a percentage of my time. I was delighted, and set off to make a big contribution and learn as much as I could about numerical relativity.

When I started at Caltech, within the first few weeks I began to hear rumors of my new advisor being difficult to work with. These rumors were never concrete and I am ashamed to say I didn't take them seriously. My PhD advisor had been rumored to be difficult to work with, which I attributed to his hands off style, but it in the end worked out very well for me with my consulting schedule. I had the mistaking impression my postdoc advisor might have a similar situation, where some people had issues due to working styles, but I might find working with him just fine. I began my work, finally getting the chance to work with numerical relativity in depth, and collaborated mainly with another postdoc in the lab. The rumors intensified, and I was taken aside by a man in the department to explain that my advisor was especially bad news, and that he had tried to bully this man out of the field. The women in my advisors group began to leave, and they would make references to him making unreasonable demands. I began avoiding him entirely, preferring to work with the postdoc. I needed mentorship to progress as numerical relativity was a new branch of science to me, and when the postdoc left for a job in industry, and I found myself actively avoiding my postdoc advisor as it became clearer the extent of his bullying and harassment, and that it was targeting women, I thought about leaving and made plans to interview and take a sabbatical position at the South Pole. My postdoc advisor was put on leave while a harassment investigation was underway. Shortly after that I left for Antarctica, hoping that the situation would be clearer when I returned a year later.

In Antarctica I ran the South Pole Telescope for 10 and 1/2 months together with a PhD student, who also happened to be a woman. It was an amazing experience being in charge of such an impressive machine in a hostile environment and I was proud and comfortable to work with my colleague to do so. Partway through the year, Caltech wrote me that they had agreed with my former advisor to move the person who signed off on my grant to another professor in the department. When I returned from Antarctica, the situation at Caltech was still complicated. My former advisor was on leave, but was slated to return. He was prohibited from working with students, but could work with postdocs. I hadn't spoken to him for more than a year, he or Caltech had apparently requested the relationship be formally severed, and by now the rumors around his conduct were substantiated findings: he had engaged in gender based harassment of several of his students. The professor who signed off on my grant did very similar research to what I did in my PhD and I worked to find common ground with my NSF proposal so that I could be more in line with his research. I then searched for a collaborator for my numerical relativity work, who ended up being a former Caltech postdoc with my former postdoc advisor, now working in the bay area. Collaboration at a distance was slow, and the numerical relativity work ground to a halt. I enjoyed and made progress with my research with the new advisor, but it wasn't what I had come to Caltech to do. I began to make preparations to leave my grant at Caltech early. I didn't feel like I could in good faith finish the research I set out to do on the grant, given the mess. I was able to fully execute on my outreach project as part of the grant, as well as publish a paper with my new advisor, before leaving the grant almost a year early to work at NASA JPL.

Without a system to report harassment, and with people being required to be silent about ongoing investigations, I felt that it became clear to me much too late that the problems students had with the advisor had nothing to due with differences of personality, and everything to do with the advisor. I wish I had known about these problems before coming to Caltech, and that afterwards I had not been so quick to map the problems to differences in personality. In reality, I should have switched research topics and advisors right away. 

My decision to leave my postdoc grant early was also impacted by my desire to start a family with my husband, who happened to have been the man who warned me my advisor was a bully. At this point I was in my early thirties, and my schedule dictated in another 2 to 3 years I may want to apply to a similar opportunity to the South Pole where being pregnant was not an option. So the next 2-3 years were ideal for pregnancy. But if we started a family while I was on my research grant, I would be searching for a new academic or industry job shortly before or after giving birth, and my maternity leave situation would be nebulous. My research grant offered unpaid leaves of up to three months, or the flexibility to work from home (some women did this throughout their maternity leave) at a self-decided pace while getting paid during this period. However, given how far I was behind on my grant work already, I couldn't see taking additional time from research as being possible, nor could I see going back full time to research days after giving birth as being practical. So I applied for and found a job at JPL. JPL has the advantage in that for many family leave options in the US you have to be working at an employer for more than a year to receive maternity leave outside of medical disability (approximately 6 weeks), but since JPL was managed by Caltech I would count as working there for more than a year were I to get pregnant, so I would be eligible for 12 weeks of leave (a combination of a percentage of my pay and unpaid leave) on top of approximately 6 weeks of disability. I went to my new postdoc advisor, shared with him the opportunity, and we agreed I would take it. I negotiated to keep my office at Caltech and to continue to finish some of the work we had begun since I returned from the South Pole. 

I had no idea how long it might take to get pregnant, but we began to hope our family dreams might come true within the 2-3 year window. Shortly after I started at JPL what my husband and I suspected was confirmed, I was pregnant. I chose to wait until the fetus was considered less likely to miscarriage to share the good news with my new boss. By that point, my husband had also switched jobs to join JPL as well. At first the HR told us that my husband and I would need to split a portion of our partially paid leave with each other because we shared the same employer. This is indeed the letter of the law in California. Later, we were told that Caltech had changed its policy to go above and beyond the law to allow us each a full 6 weeks of partially paid baby bonding leave. My husband and I sketched out a plan where we would each take 80 days off over the course of the baby's first year. My husband wasn't entitled to his leave until a year after his start date, as he didn't work for Caltech in his job immediately previous, so the majority of his leave would have to be taken after the baby was about 6 months old. However we were both glad we had stable jobs with leave provisions, as well as a wonderful daycare that we were admitted to for 3 days a week when the baby was 6 weeks or older. JPL is very diverse as far as the physics academia or tech industry standards go with respect to age, race, gender, and more, and is known to be highly supportive of families. We have found this to be the case thusfar. My group at JPL has a female principle investigator and a male manager, although the team has been for most of my time 100\% male otherwise. However, I work with a variety of missions, and most of them have heavy female representation. One particular mission I work with is majority female. I recall a meeting with 8+ women where a single man walked in late. I didn't not know who he was, so I introduced myself, he said his name was Guy, and I said "oh the token Guy." It's an honor and often relaxing to work with majority or all female teams, and one I have enjoyed rarely in my career.

I work out of my office at Caltech on research projects every other week now. Science will always be a part of my life. I have 13 refereed publications over the past 11 years, with more than 5000 citations in multiple academic disciplines. I can also publish as part of my job at JPL. I have a book the in the works, as well as several papers underway. My story became much more complicated and nebulous due to the intersection with male bullies, harassers, or other issues, but I have always found a creative way to make that into some sort of positive outcome. I hope I have learned more along the way so that I can be a better ally to those experience harassment. 

{\bf 
Anuja De Silva discusses her experience on single sex education in highschool in Sri Lanka and in the US at Mount Holyoke, graduate school at Cornell University, and industry work at IBM. She is also the mother of two children born while she was working in the semiconductor industry.}
In Sri Lanka, I attended a girls only high school where out of 400 in the graduating class more than 50\% are pursuing a career in the STEM fields.  

I then joined Mount Holyoke. It is the first of the seven sister colleges to be established in 1837 when Ivy league colleges were male only and still continues its mission for leading the way for women in STEM. From 1966 to 2004, according to the NSF’s Survey of Earned Doctorates, Mount Holyoke graduated more women than any other liberal arts college who went on to get U.S. doctorates in the physical and life sciences (356 and 109, respectively). This puts Mount Holyoke in the top 2 percent of all colleges and universities--even major research universities with at least double the enrollment and faculty. Among all colleges and universities, Mount Holyoke ranks eighth (tied with Stanford and Wellesley) in the number of graduates who earned U.S. doctorates in physics from 1966 to 2004; ninth in chemistry; and sixteenth in biology.  Mount Holyoke also leads with its commitment to minorities. From 2000 to 2004, Mount Holyoke produced more international (non-U.S. citizen) female graduates who went on to receive U.S. doctorates in the physical and life sciences than any other college or university. Twenty-three MHC alumnae received U.S. doctorates in life or physical sciences, compared with 21 women from the University of California-Berkeley, 19 from Harvard, and 17 from the Massachusetts Institute of Technology \citep{MHC}.  

I was among the 23  Mount Holyoke alumnae who went on to receive U.S. doctorates in life and physical sciences from 2000 to 2004; 22 of them are minority women, the highest number along first tier liberal arts colleges in the United States. As a minority Mount Holyoke graduate from class of 2004 who earned my PhD from Cornell University in 2009, I’m proud to be featured in this statistic and still continue to work in the STEM field almost 10 years later.

The chemistry and chemical biology department of Cornell boasts more women than the physics department in keeping with the trends across universities. In 2003 the incoming class was about 35\% women and the numbers stayed consistent through graduation. The number of women in the department played a significant role in improving the graduate school experience for women. Friendship and support are key criteria that help  minorities succeed. I was fortunate to be a part of a research group led by an advisor who has consistently supported women graduates (our group was always 25-30\% women; 5-6 personnel) and hence kept faith in me after I ended up with unsatisfactory performance on certain classes that I had no study group support for. It was a turning point for me that bolstered more ambition and focus for me where as it could have easily turned into my giving up on the path towards a doctorate. During my thesis work, I learned that perseverance is key as I struggled for the first few years to attain meaningful results. But the support of my group members along with opportunities to interact with industry enabled me to stay motivated and focused. My graduate research was funded through an industry consortium through the Semiconductor Research Corporation (SRC), which included opportunities to attend conferences and well as an internship at IBM. This internship led to the start of my career as a post-doctoral associate at the IBM Almaden Research Center.

Working in the research and development area in a major tech company, diversity and inclusion of women is seen as a business imperative. During my tenure at IBM I have seen improvements to maternal leave policies and work life integration policies that enable both men and women to share the workload of child rearing. Due my background in women’s only environments, I’m more secure about my potential and contributions in a male dominated industry. But unlike in a women’s ecosystem, I’m also keenly aware of how much more assertive I need to be and how actively I need to seek recognition and promotion. As I enter the mid-career phase (around 10 years post graduate school), I’m faced with the lack of women in technical and leadership roles first hand.  Women constitute about 25-35\% of these roles in most leading tech companies in the United States \citep{techcompanies}. As a mid-career woman my greatest challenge I face today to is to beat that statistic and continue my prominence in my technical field.

I do notice an active effort in corporate recruiting to emphasize the pipeline of women entering corporate research roles. Currently I’m one of the senior technical staff who is also on a project lead role within my group. IBM is similar to other tech companies where the female technical staff is around 25-30\%.  Several initiatives in our division have gained management support and continue to bridge the gender disparity. They include gender bias training for all employees, an yearly all-day event that focuses on women’s empowerment, regular round tables with senior women engineers, and special assignments where women can spend up to 20\% time learning a new skill/working on a project with a mentor outside of their regular role. I have been in my current group for the past four years and I have seen an increase in contribution from women engineers in papers and patents published. As a senior woman in my group I have continued to seek out and engage women engineers and we have a high 50\% or more contribution from women in our published work. The women engineers also represent a variety of minorities (about 50\%).

Since my pregnancies were during my tenure in industry, I was eligible for paid maternity leave. With both children, I opted to stay home for 10 weeks post cesarean delivery and ease back to work. I was fortunate for the work life integration policies at my company which enabled me to take time off for doctor’s appointments and sudden child care emergencies. While I enlisted help from my mother, I also invested in paid child care, which was about 15-20\% of my salary. My childcare arrangements during work and work travel for conferences have been incurred on my personal income. I have considered them a longterm investment in myself and my family. I believe in consistently trying to find the best fit for my child as well as for my career.  The childcare costs tend to be the highest between ages 0 and 5. When children turn 5 in the United States, they start kindergarten. At that point parents who work full time will still incur costs of additional child-care at the beginning and at the end of the work day since a full day in school is only about 6 hours.  I am part of a dual income family. This has made a big difference in my ability to have two children relatively early in my career. For single income families, child rearing is a bigger financial strain. My current job also provides health benefits, lactation facilities and counseling which are necessities for working mothers.

\subsection{Maternity in STEM: from Marie Curie and Mileva Einstein to Today}
\label{Sec9}
Children are still seen as a career advantage for men, and a career killer for women in both industry and academia. Studies find that having children reduces a woman's ability to be a top earner and that family size has no impact on any of the labor market outcomes for men \citep{Cools2017}. Among tenured faculty about 70\% of men are married with children, while 44\% of women have families \citep{mason2004marriage}. To mix having a career and a family, female employees need adequate maternal leave, which is considered to be at least 40 weeks \citep{ruhm2000parental}, and access to childcare. Instead, in industry, women in the 25-45 year bracket are hired on temporary contracts that terminate if they chose to have families \citep{Cools2017}.   

%The lack of support forces more than half of the female workforce in STEM to quit mid-career. The ones who try to stay face the choice of not having children or wait until their career stabilizes when it might be too late.  While adequate maternity leave policies and the availability of childcare will indirectly decrease the gender gap, their primary role would be to enforce basic human rights and support  employees in being productive without feeling guilty that they have neglected their children and/or forced their relatives into sacrifices. 

Many women who have shaped our world have had and raised children.  Marie Curie raised two daughters -- her eldest won the Nobel Prize; the other was a bestseller author and journalist with a leading role in UNICEF. Marie Curie, her daughters, and sister were leaders in a male-dominated society and excelled. They succeeded in spite of the society they lived in. Marie chose to not only work hard, but to share her life with a man of outstanding moral uprightness. Pierre Curie was already established in his field, and did not conform to recommendations. He went as far as refusing the Nobel prize until the award included Marie. Marie Curie relied on the support of her father in-law to raise her daughter while she worked in the laboratory. Short maternity leaves and lack of available childcare on campus, force women scientists to rely on similar solutions today. Yet the extended family support system is shakier than in the past. Retirement has been pushed back, and if a grandparent is healthy, they will often work themselves instead of carrying for grandchildren. People also choose to have children later when their parents may be too old to help and in geographic locations that are far from where their family is. 

Albert Einstein and Mileva Maric were graduate students when they started their family. They were also both immigrants. Z\"{u}rich's Politechnic institute did not support them. Einstein received his PhD, but could not find work within the university. After searching for work for a long time, he was hired by a patent office. While Albert searched for work, Mileva returned to Serbia where their daughter Lisserl was born and is believed to have died or been given away a year or two later after she became ill. Mileva never received her PhD. Today, infant mortality is still correlated with lack of maternity leave and stress during pregnancy. Later Mileva worked with Albert to produce relativity while receiving no credit for it, and was quoted for saying that they were but one stone, and that with fame often one takes the pearl and the other the shell \citep{Mileva}. 

While Mileva and Marie died a long time ago, the lack of support for women who hold temporary positions  (postdoctoral and doctoral level) continues today. When they have families, many still have to quit science. Some switch careers or find ingenious work arounds with the help of extended family. To avoid discrimination and absent support, others postpone having children for a career that often does not materialize and can end up having to go through expensive and painful infertility treatments. 

Below we quote a series of prominent scientists regarding motherhood before tenure. The Head of the Gender Equality office at the Swiss national science foundation writes "I am myself a mother of two kids and left research for this reason."  A colleague with four children who is a prominent scientist at an ivy league university in the US comments "I have never had a paid maternity leave, except for my first child in which case it was not official. My adviser was extremely supportive and I keep thinking of him with the highest respect. For the other three kids I was in transitioning between cities/positions and planned so that the break would not show on my CV. Financially it was definitely negative." She succeeded to stay in science because her parents and in-laws flew from Europe to the US to take turns in providing child-care. 

Another colleague was cautiously optimistic "Things are getting better slowly in terms of maternity leave both in Europe and the US, but this does not mean perceptions in physics with our male colleagues are changing as fast. I waited to have a baby after I was 35 and I wouldn't recommend this choice (coupled with the risk of health problems and fighting against your biological clock). The main reason was that my husband and I were doing long distance [commutes]/not in stable positions" (excellence fellow in the Netherlands). She also had to enlist the help of extended family, who flew thousands of kilometers to provide it, until her child was eligible for daycare. 

A Cornell PhD who has returned to India with her husband where he obtained a professorship while she struggled on soft money writes " I have a 3 year research grant from the government, but after that things are again a bit up in the air. The money for this 3 year cycle has also not been sanctioned - so all my work currently is unpaid ! Pay or not, academia is pretty ruthless so, one has to keep working" (she had a 7-months old daughter at the time). 

The subtle pressure in STEM is to dedicate one's life to work, and to feel guilty when taking time away from what's considered to be one's life-long passion. This leads to postponing having a family sometimes indefinitely. When such pressures are defied, it can be assumed that the person will quit the field unless they are already established and that they cease to be a worth-while time investment. The same can happen when an illness appears when the person does not have a permanent position.

So, if the decision to have a child is made, the tendency is to hide the pregnancy for as long as possible, and to avoid mentioning it until after a job offer is made when applying for jobs. The child will then be planned between jobs or assignments if at all possible so that the pregnancy does not show on one's CV. Beyond this, the prevalent attitude is that funding maternity leave and providing adequate day-care will not increase the number of women in the STEM workforce. It is accepted that maternity leave and day-care on campus should, in theory, exist and be available, but also that such services are not available on most campuses, and that there is no need to hurry and make any changes to provide mothers with adequate support because they will not matter. This attitude, while far too common, is, of course, faulty and endangers lives.

While it is, at times, possible to exchange work duties with colleagues, or plan children in vacations, this should not be the norm.  Studies find paid maternity leave is not a luxury. It is a basic human right that can be life saving for both mother and child. The lack of antenatal leave has been associated with a three-fold increase in the risk of pre-term delivery and has been observed the have similar effects on the birth-weight as smoking during pregnancy \citep{ceron1996risk,del2012intrafamily}. Mothers who take short leaves are up to four times less likely to initiate breastfeeding when compared to those who do not work \citep{huang2015paid,baker2008maternal}. Paid leave for mothers is linked to increased breast-feeding rate and returning to work is cited as the top reason for breast-feeding cessation \citep{schwartz2002factors}. The American Academy of Pediatrics says infants not breastfed face more than 3.5 times the odds of Sudden Infant Death Syndrome (SIDS) mortality when compared to exclusively breastfed babies \citep{eidelman2012breastfeeding}. Furthermore, paid leave has long-term benefits that go beyond survival of the child, and benefit the child's long-term development and leads to higher achievements \citep{berger2005maternity,carneiro2015flying}. It is noteworthy that paternity leave is linked to an array of long-term benefits as well \citep{huerta2013fathers}.

Many scientists are immigrants for at least some part of their career and thus not eligible for standard maternity leave and child-care support offered to citizens, which puts them in a particularly vulnerable situation even in countries where such leaves are the norm. They have to rely then on company/university-specific policies that also provide full support only for permanent employees. Graduate students and postdoctoral scholars are temporary employees with limited rights that end with their contracts. Yet graduate school lasts an average six years in the US and four years in Europe. The postdoctoral period can last another ten years. This system leaves men and women in their twenties and thirties unprotected for maternity, paternity or illness. In companies, when policies for paternity and maternity leave exist, their applicability depends on the project the person works on with successful employees that are difficult or impossible to replace being constraint to have less time off.

In the academia, most universities do not offer an adequate support system to mothers until they become professors \citep{2018Report}. This often happens when they are in their forties, and then it might be too late to start a family.  Even for permanent staff, obtaining maternity leave can be complicated. For example, in Penn State University, maternity leave comprises of accumulated sick days.  Furthermore, having to wait for support until one has a tenured or tenure-track position, requires to postpone having a family. It also often means choosing between having a career that is likely to not work out since a very small number of postdoctoral scholars become professors and having a family. This adds to the toxicity of the environment.  For students and postdoctoral scholars, the work schedule is flexible and an understanding advisor will try to navigate the system to help their postdoctoral scholars and graduate students take advantage of various workarounds. %Unfortunately, the attitude towards maternity coupled to the lack of supportive services for child rearing, forces women to either postpone childbearing until their career stabilizes, which could be never, or jeopardize their career. %Since many scientists tend to be immigrants, they will not have the right to maternity leave or sick leave even in countries where such leaves are the norm. 

The few specific programs that encourage women to return to work are done in a proforma way. For example, the Horizon 2020 program of the European Union included the Marie Curie Career Restart (CAR) Program to decrease the gender gap and help scientists who were out of the field come back to work. In physics, it had a lower success rate than the regular program; so experienced staff advised applicants against checking the CAR box when they qualified. Scientists were told this was because all the money came from the same pocket since the CAR program had no additional funds. 

Switzerland funds programs that aim to fix the leaking pipeline to encourage the hiring of more women faculty. However, departments and advisors neither have the requirement nor are encouraged to report pregnancies to human resources or provide maternity leave beyond the end of the contract. When asked Human Resources argue that they try to provide a case-by-case workaround when the pregnancy is reported early enough to their office. Advisors are generally supportive. This means they tacitly accept a lesser presence in the office during the contract while pregnant, fewer research results, and perhaps a longer time to graduate. Still, at the end of the contract, women scientists can end up in their home country unemployed, heavily pregnant or with a young child, and without the right to paid leave or even health insurance. Unfortunately, in these circumstances, the chances of staying in science are minimal. The women who are under-contract and have children, have to plan and get on the waiting list for the often extremely small day care on campus sometimes before their pregnancy even begins. The insufficient child-care facilities are a problem that is common to university campuses across the world \citep{2018Report}.

We note that in countries like Sweden that have the longest paid maternity leave in the world (16 months), the number of women awarded faculty positions is still of the order of 20\% over all fields. This is because of sexism and institutional bias \citep{Sweden}.  While maternal and paternal leave like in Sweden should be the norm, such policies alone will not directly reduce sexism. 

\section{Behind the STEM Gender Gap}
\label{causes}
 As shown through our stories, even though we are well educated and were privileged to have educated parents who valued and supported us, each of us almost did not make it. Even though we each worked hard, it seems we have succeeded through a series of 'work-around' situations, which were enabled by supportive colleagues, advisors, and secretaries, who were willing to go beyond their job description to help, while many others who were equally qualified were lost 'through the cracks'. 
 
 Below we briefly review the accepted reasons behind the gender gap, and then mention interventions that could connect different educational levels and enable women and minorities to stay in STEM. We note that we do not directly address every day sexism and institutional bias.  They are, however, present as can be seen from our stories. Data shows that women have to have more than twice as many publications as their male colleagues to receive the same competence score as male applicants \citep{christine1997nepotism}. Similarly, due to gender bias, grants written by women have to have stronger merits to be funded \citep{bornmann2007gender}. Due to institutional bias women are passed over for promotions, while more senior male collaborators are praised for their work \citep{rickard2015slower}. They also publish less and are underrepresented in both first and last author positions and as authors of single-authored papers \citep{west2013role}. Their careers are on a slower track because they are likely to work part time when they have families. Women are also likely to be the partner who compromises and takes a job below their level of expertise so that they can be in the same city with their spouse/partner \citep{rickard2015slower}.  A bias against women is exhibited by both men and women when evaluating identical information with different names \citep{christine1997nepotism,bornmann2007gender,rickard2015slower}. To reduce bias, intervention programs that increase the number of women have to go hand in hand with the education of both men and women against gender bias with particular emphasis on those in leadership positions. 
 


\subsection{A First Problem: Attitude towards Girls and Women}
\label{Sec2}
 Data shows that gender disparities arise from the attitude of students towards learning \citep{organisation2015abc}. In all countries, girls lack self-confidence in solving math and science problems \citep{organisation2015abc}. This is likely because, for centuries, a woman received positive reinforcement from an early age only when being passive and nurturing. Her curiosity, analytic mind and ability to take things apart and put them back together were not rewarded the same way \citep{signorella1987children, francis2010gender, organisation2015abc, KSmith}. Today, when one goes to a toys store, there are clear indicators that certain toys are for boys, and certain other articles are for girls. From labels to colors one is informed what is for boys and what is for girls. Construction and mechano-spatial toys, trucks, cars all scream “boy”. T-shirts that encourage leadership roles are generally in the boys sector. Only boys can aspire to be batman, superman or spiderman. Girls are mostly expected to dress in frilly and revealing clothes and aspire to be saved by, or if they are extremely lucky, marry the superheroes \citep{graff2012too}. This stereotyping is encouraged by toys \citep{francis2010gender,KSmith}, movies \citep{bleakley2012trends}, and books \citep{caldwell2018hairdressing, barrs2000gendered,hamilton2006gender,frawley2008gender}, and by the larger society as well \citep{KSmith, witt2010self}. It makes children feel unfit if they are attracted to articles from another category \citep{KSmith,cvencek2011math}, and it makes girls unprepared for the STEM work in which spatial reasoning plays a crucial role \citep{taylor2013think3d,casey2015longitudinal}. 
 
 A lot of learning happens through making mistakes and then reworking through and correcting them. People and societies have been more indulgent to males being exploratory and breaking objects before fixing them \citep{francis2010gender}.  Women are thus more diffident in exploring because they feel they will be chastised if they do not obtain the right answer on the first try \citep{organisation2015abc, KSmith}. This attitude  results in a significant gap in confidence between women and men.  If women are rejected by an employer, they are 1.5 times less likely than men to apply to another job opening by the same employer \citep{brands2017leaning}. Women are generally better at group work because of their attitudes of consideration \citep{Yardi}. However, if the group dynamic is such that the women and men take only the men in the group seriously \citep{Guardian}, it has serious consequences for the creative expressions of women \citep{pollack2013there,corbett2015solving}.
 %Women from all girls school were observed to be more likely to study science and mathematics and performed better in these subjects \citep{organisation2015abc}.

\subsection{A Restrictive History and the Tendency to Maintain the Status Quo}
\label{Sec3}
Men have had a much longer history of exploring their passions and interests, of being able to have a say in how life has to be lived, how science is presented, and in being able to vote than women. There was a time when, in the US, women were barred from certain work post marriage \citep{rindfuss1996women}. Women’s suffrage happened as late as 1920. Women authors had to write under male names (e.g., George Eliot) because women were not allowed to publish. Such restrictions made it hard for women who were competent to grow and reach their potential. During wars, injured men needed care and many women were conscripted into nursing. This history has led the way to more women being present in biology and health care fields than in other STEM areas \citep{amott1996race,drew2011women}. In general it is not just men who prevent women from succeeding. Women who maintain the status quo are lauded by society and also restrict women who think differently \citep{rudman2012status}. Women must be made aware of their own role in encouraging the success of other women, not underestimating women, and in exploring different ways to promote knowledge for different learning styles. Additionally men must be made aware of the advantages of history that have contributed to their success so that they also become enablers of success independent of sex, color or nationality.

%It is thus important that both women and men are aware that the lack of female role models in STEM and other fields is not because of incompetence but because of a restrictive history.


\subsection{Isolation, Stress and Vulnerability}
 %‘https://ngcproject.org/statistics. 
 In STEM, because of their low numbers, women and racial minorities stand out. Their interests and abilities, and even the way they dress and the way they behave is noticed and judged. This scrutiny adds additional pressure to fields where very few succeed and pushes out the few who almost make it to the very top of their professions \citep{pollack2013there, 2018Report}. Research shows that women in STEM experience more stress than men and are more likely to be perfectionists, which helps face criticism, but it is known to increase the stress level for both genders \citep{rice2015perfectionism}. %We conjecture that the perfectionism of the few women who succeed arises as a defence strategy against the attitude exhibited towards girls and women.
 
Since they are different, women and minorities are more prone to abusive comments and actions  from colleagues and/or supervisors than white men \citep{quinn2002sexual,hill2010so, Guardian, 2018Report}. Furthermore, for both genders, any potential abuse can be facilitated by immigration status and by the very limited set of rights provided by universities for their temporary employees  \citep{report02}, which include graduate students and postdoctoral scholars. %The absence of policies regarding reporting abuse and not educating departments on how to deal with either the victims or the perpetrators deters victims from reporting because they are . 

\subsection{The Leaky Pipeline Across Nations: Most Women Drop Out After Partial Success}
\label{leaky}
Most countries suffer from the 'leaky pipeline' problem, which starts at PhD level. When averaging over 137 countries, the number of women researchers is only 28\% \citep{huyer2015gender}. World-wide,  women drop out of science and engineering mid-career. In the US in 2017, women held only 24\% of STEM jobs even though there were paid 35\% more than women in non-STEM jobs and had a smaller gender wage gap than in non-STEM jobs \citep{womenSTEM}. Then, out of 41\% women scientists, engineers and technologists on the lower rugs of the corporate world, 52\% of them drop out to switch to less technical fields in their mid to late thirties \citep{hewlett2008athena}. 

Some women transition to administrative positions inside their institutions while many quit altogether while believing they were not good enough to succeed, which is an attitude that is not conducive to maximize success elsewhere \citep{huyer2015gender, 2018Report}. In academia, talent is lost at the end of the PhD or after the first postdoctoral position \citep{2018Report}. Most women leave their field of expertise in their mid and late thirties likely due to gender bias, which ensures women have to be more qualified than their male counter-parts to qualify for the same positions and promotions, coupled with the lack of maternity leave, which makes motherhood incompatible with successful STEM careers \citep{corbett2015solving, huyer2015gender}. Most scientists are foreign born, which means they have a restricted visa status that is based on a temporary contract that often come with no provisions for maternity/paternity leave or illnesses and hence have fewer rights than nationals \citep{report02}. In the US, up to 80\% of full-time  graduate  students  in  electrical  engineering  are foreign born. Most other STEM fields have over 60\% international students \citep{anderson2013importance}. However, even in more liberal countries like Sweden, everyday sexism and gender bias cause a large gender gap \citep{Sweden}.

%In the US, approximately 40\% of science and engineering post-graduate students in the United States are foreign-born \citep{DegreesUS}. Similar percentages are observed in countries in Western Europe over all fields. In Switzerland, international students constitute 29\% of masters students and 54\% of doctoral students \citep{Degrees Switzerland}. In Germany in 2017, 34\% of bachelor degrees, 47\% of master degrees and 10\% of doctorate degrees were granted to international students, and the number of international students continues to increase \citep{DegreesGermany}.

%Scientists who do succeed in obtaining STEM bachelor degrees and hope to attain an academic role are expected to work outside their home country. In the post-college training period that can last 15 years or more their rights are restricted by their immigration status. Obtaining a doctorate degree can take five years or more, and the postdoctoral period extends to the mid (and sometimes the late) thirties. Doctoral students and postdoctoral scholars are temporary employees with the temporary status lasting many years along with a very limited set of rights. 
%Similar percentages are seen in Scientists who are immigrants have a restricted visa status and do not have the same rights as regular staff during maternity/paternity leave or illnesses.   More generous policies to maternity leave and childcare are awarded to women who have tenure-track or permanent positions, but such positions are often reached past the child-bearing years. Since the support net is not there when it is needed, 

In Europe, women are underrepresented in STEM with the annual growth rate of female graduates being smaller than that of men in all fields in STEM \citep{She2018}, which implies that parity will not be reached in the foreseeable future. Finland is one of the countries where gender parity is reached at college level in natural sciences, mathematics and statistics university degrees  \citep{FinlandStats}. Yet the number of women professors is 24\% across all disciplines even though for the last 30 years, 50\% to 60\% of university graduates have been women. The numbers vary across disciplines with women in physics comprising 25\% of graduates and only 7\% of professors \citep{banzuzi2013women}. In Switzerland, ETH Z{\"u}rich had 31\% female students in 2017 while only 13.9\% female professors with the number of women holding full or associate professorships being 12.1\% \citep{schubert2017gender}.
 %So, even when parity is achieved at college level, the climate in science and engineering forces women and minorities out of their field of expertise after they obtain a PhD.
 
  \cite{stoet2018gender} find that countries with greater gender equality have fewer women by percentage in STEM than many countries with higher gender inequality. It is clear that apparent gender equality in every day life does not mean a gender equal STEM culture.  Subtle biases add up and result in women reporting lower job satisfaction and productivity compared to men and in more women than men leaving their field of expertise \citep{settles2014women}. Women in developing countries are better represented in STEM degree programs than in developed countries \citep{India2017}.  In the 2016-2017 academic year in India, women obtained around 47.6\% of science degrees, 42.4\% of IT and computer degrees, and 28.4\% of engineering and technology degrees \citep{India2017}.  However, they still expressed lower confidence, less assertiveness, and the unlikelihood of expressing an answer for fear of being wrong \citep{Yardi}. Even though women held almost half of the science degrees, only 20\% of the tenured-track faculty in physics were women, and of these only 12 women (compared to 197 men) are Fellows of the Indian Academy of Sciences  \citep{WiredIndiaFaculty}.

%Since most women leave their field of expertise in their mid and late thirties, the lack of polices that ensure maternity leave and facilitate the return to work does play a role when leaving demanding careers \citep{rangel2018factors} along with the attitude that is naturally exhibited towards women and minorities \citep{national2018sexual}.  

\section{How the US compares to India and Eastern Europe}
\label{Sec4}
For every 200 students getting degrees in India and the US, 52 Indians and 26 US residents obtain STEM degrees. Of these students, in India, 32 will be male and about 18 female (male:female ratio of 2:1), while in the US, 21 will be male and 5 female (4:1 ratio) \citep{STEMDegreesByCountry}. We conjecture that the pressure to be in STEM in developing countries is from families understanding the job market and having a major role in their children’s decisions and is not correlated to how comfortable women feel in the field. In the U.S. the career decision comes from individuals who seek out their comfort zones based on the cues they get and from how their aspirations project on the social environment around them. 

\begin{figure}%[ht]
  \centering
  \includegraphics[width=10cm]{PopulationPieChart.pdf}
  \includegraphics[width=10cm]{PieChartSTEM.pdf}
  \caption{a) US population in 18-64 age group and b) STEM population in the same age group. It can be seen that the percentage of women is always comparable to that of men in the general population, but the STEM population is overwhelmingly male even for Asian minorities where both gender are pressured towards science and engineering. The data is from Guterl, F, 2014.}
\label{fig:pie}
\end{figure}

A recent study \citep{escueta2013women} found that while women in India experience no bias in school, they experience bias in the larger society. They also hypothesize that there may be biases that are ignored by women, but that this is unlikely. The lack of overt biases may reflect a change in attitude from when one of the authors (J.B.) attended school in India when engineering and technology were considered to be 'smart fields' with boys being suited for these fields (These attitudes are highlighted earlier in the paper.) Nevertheless, the pressure on boys to join lucrative fields like computer science and engineering, irrespective of interest or inclination is higher than on women. This is because they are seen as the providers for their family. They react to this pressure through increased competitiveness and attempts to eliminate any competition, which makes it more difficult for girls and other minorities to fit in. The pressure to conform to femininity standards and be perceived as non-aggressive shapes the behavior of women and forces them to leave STEM positions and avoid leadership roles.
 
 In the US, white women hold the highest percentage of the STEM workforce relative to women of other origins. Figure \ref{fig:pie}a) shows the distribution of the US population of ages 18-64 in terms of race and gender, while Figure \ref{fig:pie}b) shows the distribution within science and engineering.  White females are 32\% of population, and also under-represented at 18\% of the STEM workforce. The probability of a black woman to study STEM is less. Black females are 6.5\% of the population and only 2\% of the STEM workforce. On the other hand, Asian females comprise 2.7\% of the working population and 5\% of the STEM workforce, and Asian males are 2.4\% of the population (close to the Asian females population percentage) and make up 13\% of the STEM workforce -- a significantly larger percentage of the work force than Asian females. So, while in the US people of Asian origin are overall more likely to enter the STEM workforce than white people, the push towards STEM is more effective in the male population than in the female population \citep{guterl2014diversity}. 
 
In the European Union (EU), women comprise only 17\% of the information, communication and technology (ICT) students. However, in Eastern Europe, the ratio of males to females is closer to parity in STEM fields. Among the member states of the EU, Eastern Europe has the highest number of women studying ICT with the highest number of women being in Bulgaria (34.4\%) and Romania (29\%) \citep{EuStats}. Unfortunately, the environment in Eastern Europe is still highly corrupt.  Across science and politics some women make it to the top, but not the ones who are best at what they do because they pose a threat to the existent order. In Romania, we found no faculty who obtained degrees from universities abroad. 

%In Romania, Laura K{\:o}vesi, a talented prosecutor, was removed from office and threatened with jail time, while the first woman prime-minister is a known puppet for the party leader.

%The attitude that women should place men first or be considered a threat to society and to the well being of their family still prevails. Females are often expected to be emphatic at work while listening to jokes about beautiful secretaries, and be the party that buys the necessary things at home from their income alone, while the man of the house enjoys his at a bar. Romania has its first woman prime-minister, but she was selected to be a sort of puppet for the male leader of the party to push along. Children watch her speeches in school because she is funny when she exposes her lack of knowledge. Across science and politics some women make it to the top, but there is little tolerance for those who are best at what they do like Laura K\:{o}vesi and pose a threat to the corrupt environment.

Overall, the message that women and minorities are not suited for STEM appears to be more effective in cultures where women are free to choose a career in which the employees can be comfortable. In countries like India and in Eastern Europe, STEM education is seen as a sure path to a reasonable income. Families then pressure both men and women to succeed in these fields irrespective of other environmental factors. Under pressure, discriminatory factors can become something one has to deal with.  The primary and secondary education is focused to this end. This idea is reinforced through school uniforms and standard classroom curricula. Although the society still has a clear bias that favors men, more women make it into STEM fields.
 
The expectation that men do more valuable work than women, and hence are more valuable for the society is far-reaching and goes beyond STEM statistics. In India and China, the sex ratio is skewed towards men, and is still increasing \citep{YouthIndia,hesketh2011consequences}. Parents want fewer children while ensuring there is a son.

Cultures with group dynamics react differently to the same pressures than cultures which are more individualistic. In the former, the family pressures to succeed in these fields irrespective of other environmental factors makes those factors something to deal with. However, the existence of such factors in the first place, creates an intimidating atmosphere that can kill creativity and innovation. The group dynamics pushes more women to obtain STEM degrees. However, more quit without establishing a STEM career than in developed countries due to cultural expectations and sexism. 

In India, women make up only 14\% of scientists, engineers and technologists employed in R\&D institutions compared to the global average of 28.4\% \citep{TimesOfIndia}. Unfortunately, women fail to convert their degree into a career because of family pressures, which appropriate maternity leave might partly relieve, institutional bias, sexism, and the biological clock \citep{bal2004women,TimesOfIndia}. In 2017 90.8\% of U.S. patents were granted to men versus 9.2\% to women. The numbers of patents granted to women is still very low even though a positive grow trend of about 18\% is reported: the percentage went up from 7.8\% to 9.2\% in the past 10 years \citep{womenPatents}. It seems that the push towards STEM irrespective of the environment is not always desirable because it suppresses higher end creativity and innovation while raising the average STEM competence \citep{nager2016demographics}. 


\section{Interventions in Education that Decrease the Gender Gap}
\label{Intervention}
\subsection{The Biggest Challenge: to Reverse the Anti-science Attitudes}
\label{in1}
The pathway to interest in any field is often through activities one grows up doing. From clothes to toys, science and leadership are still portrayed as a boy-man-interest. The stereotyping goes beyond toys. Boys often deconstruct gadgets, reconstruct them, sometimes incorrectly and even break them. This behavior actually helps them learn and explore and is vital to the growth of STEM skills. Such behavior by a girl will often be discouraged. She will be admonished and will learn to stop these explorations as inappropriate. This continues in school, college and beyond. %Some gender specific books have appeared that recount the adventures and successes of women, but they are by far not enough to change such fundamental stereotypes.

The conjecture that attitude and not aptitude is causing the gender gap has been validated by various studies  \citep{penner2015gender, leslie2015expectations, organisation2015abc}, but no steps have been taken to change the attitude. Yet, most children’s toys have clearly defined gender roles inherent in them. These have to be removed to allow children to choose the roles they play.  A study of video games shows that despite an increase in female characters, games still depict them often in secondary roles and sexualize them much more than their male counterparts \citep{lynch2016sexy}. Shops, games and shows have to portray women and girls as valuable assets of the society, and not as sexual objects. Furthermore, the role of the media, which is led by educated individuals, should not contribute to the objectification of women and should not  pressure women to conform to some fake ideals.
 

%, and the culture of blame that is perpetrated, causes self-doubt, which is detrimental to individuals and to careers across all fields. The message women and girls receive is that they have not succeed because they are lacking, they are not brave enough and not strong enough to make into the 'hard' sciences, and that, instead, 

To change the attitude towards women and girls, the early learning of girls has to be more supportive to enable them up to stand up to the rigors of male dominated fields. Starting in pre-school all children could be exposed to a variety of science books and experiments, while allowing them to lead towards more depth through asking questions.  Stores could label most products without assuming a gender. Children products could come with recommendations based on age, size and interests exhibited by the child to allow them to pick in an unbiased manner. Women must be taught to take risks, make mistakes and be bold similar to how boys are typically encouraged. For this, mentoring has to happen from kindergarten age onward and continue to evolve with age. The love of learning and the growing of skills needs to be a continuous focus especially post-graduate school. Though attaining the PhD may seem as the end goal, it is just the beginning of a professional STEM career.
\subsection{Evening the numbers in high school}
\label{6.1}
\begin{figure}%[ht]
  \centering
  \includegraphics[width=10cm]{WomenMITPhysics.pdf}
  \includegraphics[width=10cm]{MECoursesMIT.pdf}
  \caption{The percentage of women at MIT in a sample of a) Physics b) Mechanical Engineering courses. In physics, parity is only reached in the first two introductory courses, while the rest of the courses held a mean of 20\% women in 1996 and 22\% women in 2016. In mechanical engineering, in 1996 the mean was 26\% women, but in 2016, after a successful intervention program and after a modernization of the curriculum, the same courses have a mean of 43\% women.}
\label{fig:MITWomen}
\end{figure}
The number of girls in science classes in highschool has increased since the 1990s due to intervention programs like AAUW, the Girl Scouts, Girls Inc., Tech Bridge and Girls Who Code,  that offered after school STEM programs and scholarships for girls \citep{highschoolNumbers}.
 While the increase in the number of girls studying science and math clearly shows that the gender gap is not due to innate abilities of men versus women in these fields, it has so far failed to lead to an increased female STEM enrollment in college. However, schools like the Massachusetts Institute of Technology (MIT) and the Harvey Mudd College have shown that gender parity can be reached at college level.
 
 Since intervention programs that link schools and colleges are so successful, more such programs are needed. Graduate students in universities around the world can go beyond being research tools, and be paid to spend time explaining science to the next generations.  This is currently done with volunteers in one-day workshops such as Expanding Your Horizons at Cornell University and Astrofest at the Pennsylvania State University. Similarly, the UIUC Physics Van prepares exciting physics demonstrations for children in elementary schools. These events have proven to be extremely successful, and could be expanded to University-funded semester long programs that expose children to science.  
 %programs that connect high school students to colleges, social media posts from female graduate and undergraduate students to whom high school and college students can relate and see them as proof of success and as role models, and the presence of more women faculty lead to parity at college level. 

\subsection{The Restructured Mechanical Engineering Department that Reached Gender Parity}
In a climate where women receive 19.5\% of bachelor degrees in engineering  and only 7.9\% of mechanical engineers are women, the Massachusetts Institute of Technology (MIT) succeeded to attract 49.5\% women in 2017 to its mechanical engineering program. Fig. \ref{fig:MITWomen} shows the percentage of women in a sample of 12 physics courses and 9 mechanical engineering courses from MIT. The data is taken from the interactive map available online \cite{GenderDiversityMIT}. From the figure, it can be seen that mechanical engineering department went from under 30\% women in 1996 to almost-gender balanced classes.  In physics, in the sample of courses we considered, gender parity is only reached in the first two introductory courses. The rest of the courses held a mean of 20\% women in 1996 and 22\% women in 2016. In mechanical engineering, in 1996, the mean over the courses considered, was 26\% women. The situation has improved dramatically by 2016. After a successful intervention program that increased the number of women admitted and included a modernization the curriculum, the same courses have a mean of 43\% women. It is notable that the number of male students that enroll in the courses has increased as well. %Introductory computer science courses show a similar increase in the number of women enrolled.

Gender parity was achieved through deliberate structural changes of the department that included (1) stimulating talented women to apply to MIT through on-campus visits and intervention programs like the Women's Technology Program (WTP), where talented highschool students live on campus and participate in classes and laboratories for an entire summer, (2) increasing diversity through the hiring of more women faculty, and (3) modernizing the curriculum in both content and pedagogy \citep{GenderDiversityMIT, xugetting}. The latter part should not be overlooked. Courses can no longer push students to simply acquire information that can already be found on the Internet. They need to instead focus on teaching students to understand and use available data and tools. The presence of women faculty was shown to play a crucial role in attracting other young women to the department who saw them not just as role models, but as a proof that success and a stable job can be attained in the field as a woman \citep{xugetting}. 

Another tool are student blogs that show that the campus is no longer predominantly male and that women can thrive at MIT \citep{xugetting}. They promote their experience through social media, which attracts more students from the next generations. Students participating in laboratories and classes on campus took the social media by storm when they shared selfies of the experience that lead to the hash tag \#ILookLikeAnEngineer.  They even proposed a reality in which women scientists are revered as much as actors and athletes \citep{GenderDiversityMIT}. 

\subsection{Admission and retention in graduate school}
\label{6.2}
The next stage after college is either employment or graduate school. For the latter, admissions rely on more standardized tests to select their students and top universities boast very high scores. Studies have shown that performance in graduate school is correlated with GPA and with the ability to communicate and interact. The standardized tests, introduced since the 1940s as admission requirements, were found to be better predictors of gender and color than professional success \citep{ripin1996fighting,miller2014test,miller2019typical}. These tests play a major role in increasing the gender gap by keeping women and minorities out of top schools. Some graduate schools do not require standardized tests, but recommend them. However, test scores are still the primary basis for evaluating students who are chosen for college and graduate school because they allow for fast comparisons of students of different backgrounds. Programs such as the Fisk-Vanderbilt \citep{miller2014test}, which offer a master that acts as a bridge to the PhD, employ a 30-minute interview instead of a Graduate Record Examinations (GRE) cut-off. It proved to be very successful with a  retention rate of women and minorities of over 80\% towards the PhD. 

Since the late 1990s, Cornell University has opted against applying a GRE cut-off when selecting graduate students in physics and astronomy. If the student has outstanding GPA and research credentials, low test scores are ignored.  Cornell even accepts some men and women with GRE subject test percentiles under 50\%.  In the past, these students did well in their coursework, passed qualifying exams, and graduated without problems. However, most Cornellians have high GRE scores. For their incoming class of 2018 in physics, the average percentile for the Physics GRE Subject Test as reported by students to the school is 78\% for the 5 women who matriculated, and 82\% for the 24 men. The students who reported their nationality were either White/Caucasian or Asian American. The number of students who matriculated with GRE subject test scores under 65\% is 4 or about 14\% of the incoming class. The lowest Physics GRE Subject Test score reported by a student was 42\%, and is held by a young man who comes from an ivy league school and has had a GPA of 3.92. The incoming class has average GRE General Test percentile of 90\% and 83\% for the quantitative and analytic section, respectively. Our brief analysis shows that while one can get away with low GRE scores if they have other outstanding credentials, this is not the norm at Cornell, and that the GREs are still an important part of the selection process.

In terms of gender ratio, out of the students offered admissions in 2018 in physics 23\% were women. Fewer women accepted the offer than men, and so the incoming class of 2018 in the physics department is 17\% women (data provided by the Cornell Physics Department). This is comparable to the average number of women obtaining PhDs in physics in the US, which is around 20\% \citep{apsData}. While these numbers are still low, Cornell has a high retention rate relative to other universities. We looked at data starting 1993, and found that between 1993 and 2011, the Cornell physics program graduated 54 women PhDs out of 74. This means that 73\% of women graduated with a PhD from this ivy league school, when the average graduation rate for graduate students was reported by the National Science Foundation (NSF) to be 59\% in 2008. The high attrition rate is due to the supportive environment. Roles models also play a role. The physics department has 8 women faculty (amounting to 15\% of the faculty body). In 2010, only about 15\% of PhD physics departments had 5 or more women faculty members, while 47\% of  bachelor’s-granting departments and 8\% of PhD physics departments had no women at all \citep{ivie2013women}. In addition, Cornell hosts about 4-5 women speakers per semester in their physics colloquium. They also encourage women graduate students in proposing and inviting women speakers of their choice, and in meeting with the speaker for informal conversations over lunch and/or dinner. 

In the past decade, the physics department at Cornell admitted an average of 4.7 women/year, and in the decade before that 4.8 women/year. Thus, Cornell data is consistent with the global trend for the advancement of women in science in that physics will not reach parity or increase the number of minority students unless admissions procedures are changed dramatically. 

In the US, the number of women obtaining bachelor degrees in physics hovers around 20\% since 2005. While retaining as many women from this pool as possible is important, it is not sufficient.  In the long term, attracting more children to science is needed. Graduate students love science and could be the best resource for teaching it to the next generation. If each graduate school had a requirement that its graduate students work with school teachers and teach for a semester at a school in the area, more children would be exposed to science and perhaps be molded into future PhDs. Invariably, teaching in schools would be promoted as a viable career choice, and not be looked down upon as some kind of work that only those who fail to become university professors do. In addition, more universities could have programs where talented high school students take classes and are part of laboratories for a semester or a summer. Participation in such programs could be offered for free to talented women and minorities. Such a program was proven to work by MIT's Mechanical Engineering Department. Advertising graduate programs to universities and to countries that graduate more women scientists could also increase the applicant pool.

\subsection{Retaining Talent Beyond Graduate School}
\label{6.3}
Since only a few percent of PhD holders become professors or permanent staff \citep{larson2014too}, it is important to stop seeing the degree as the ultimate goal that in itself insures success and invest more in connecting STEM graduates with potential employers via workshops on campus and through summer and semester-long programs. This is currently done only in certain experimental fields where the expertise obtained in graduate school is directly relevant while theorists and other experimentalists fend for themselves. 

The number of women in tenured professor positions is still very low with some departments having a single woman faculty member and some having no women at all. In industry, the percentage of women-headed ventures flattened at 17\% in 2012 \citep{teare20172017}, and has seen no growth since. These low numbers emphasize the need to (1) invest in intervention programs that make a smoother transition for their graduates (2) invite talented women to apply and hire more women, (3) increase the number of women mentors and potentially induce a snow ball effect that attracts more women to STEM PhD programs, and (4) to build support networks that connect existing students to potential mentors in industry. The PhD should not be seen as an ultimate goal, but as a step towards a career in STEM, where universities and companies work together to make the transition smoother. For those who do not pursue a PhD, college education has similar issues.


% For women faculty members, feelings of isolation, lack of respect of colleagues, and difficulty in integrating family and professional responsibilities are major factors in attrition from university careers. \cite{committee2007beyond}
% PhD granting departments with 5 or more female faculty: https://www.aip.org/statistics/reports/women-among-physics-astronomy-faculty

\subsection{The Oldest Intervention Programs: Single Sex Education and Historically Black Colleges}
\label{Sec7}
Another resource for obtaining gender parity  are single sex high schools or women’s colleges for undergraduate studies. They can substantially help women pursue and command STEM fields. In the absence of boys/men, the focus is purely on women’s capabilities and not contrasted against the opposite sex. Especially in cultures where boys are regarded as more valuable from a young age, having single sex schools can help to mitigate that effect in the daily lives of women and girls. %Though girls grow up with the added burden of having to be more pleasing in terms of physical beauty and subservient in attitude, they learn to be certain of their aptitude in academics due to single sex education. It also provides an environment for girls to be mentored and recognized in the absence of boys which can have a lasting effect as they grow into young women.

Historically, the role of women's colleges was to give women access to education. Before their establishment the  education of women was usually limited to homeschooling by governesses who were only affordable to wealthy families.  In 1851, the College of Notre Dame became the first women's college. Today, women's colleges aim to make women comfortable enough to feel free to build on each other without being pressured by social standards, stereotypes, preconceptions and self-doubt. Data shows that women graduating from single sex institutions take more science courses and perform better in STEM than women from co-educational schools \citep{organisation2015abc}. Many known leaders graduated from women's colleges including Hillary Clinton, the woman who almost won the US presidency, Leah Busque, founder and CEO of TaskRabbit, and Grace Hopper, the person who developed the first compiler. 
%References: https://www.brookings.edu/blog/social-mobility-memos/2017/01/19/the-contribution-of-historically-black-colleges-and-universities-to-upward-mobility/
% https://www.npr.org/sections/codeswitch/2017/03/01/517770255/hbcus-graduate-more-poor-black-students-than-white-colleges

%\begin{figure}%[ht]
%  \centering
%  \includegraphics[width=8cm]{HBCUFig.pdf}
%  \caption{The number of black students enrolled in STEM is shown for every 1000 black students in accredited U.S. bachelor's degree programs in the Fall 2008 semester. It can be seen that the gender ratio is close to parity only in HBCUs.}
%\label{HBCUsFig}
%end{figure}

Similarly, Historically Black Colleges and Universities (HBCUs)  serve the educational needs black Americans. This paper is not about racial and cultural minorities. However we do want to compare and contrast some of the issues and remediations. The first HBCU was Cheyney University. It was established in 1837.  Prior to the time of their establishment, and for many years afterwards, blacks were generally denied admission to traditionally white institutions. As a result, HBCUs became the principle means for providing postsecondary education to black Americans \citep{BlackHistory}. %https://www2.ed.gov/about/offices/list/ocr/docs/hq9511.html  
While all students can access colleges today independent of their skin color, HBCUs have far more low-income black students than PWIs, allowing them to attain an education that economic inequities would otherwise deny them.

Today, the role of Historically Black Colleges (HBCUs) is to reduce the isolation of minorities in PWIs \citep{reid2012women}. In polls on HBCUs versus PWIs, it is clear that black students on an average felt better supported in HBCUs than in PWIs. \cite{seymour2015grads} reports that 29 \% of black graduates who did not attend an HBCU said they were ``thriving in financial well-being,"  and 51 \% of black HBCU graduates reported doing so. Most HBCUs reach gender parity in STEM courses and in some female enrollment exceeds male enrollment \citep{simms2014educational}. The lack of male students is connected with the hardships they endure while living in underserved communities \citep{cuyjet1997african,white2013black}. Black people of both genders have a longer history of exclusion than white women \citep{hine1997hine}.

Public education today allows all women and men to go to public schools in all the countries the authors come from. Nevertheless, most countries suffer from an the economic bias of how good of an education one gets based on wealth. This economic bias makes some minorities be underserved beyond gender. In the case of African Americans there are the added issues of racism. The themes of being kept from reaching their potential, and being kept out are much more profound in this context. 

Movies like ‘Hidden Figures’ highlight the issues of how the achievements of African-American women have been downplayed. One way of downplaying achievement is through controlling how their stories are told.  The contributions of some women and minorities are acknowledged while emphasizing that they were exceptions in the group they represent. This technique prevents a group from achieving its full potential by reducing the competition and breaking the spirit, adding roadblocks, and making people uncomfortable. 

Formative education of women is undermined by making them constantly feel wrong about feeling interested in STEM fields. Economic bias does not target women particularly, children of economically stable families have equal access to good schools whether they are male or female. However, African-Americans and Hispanics have a lower economic base and thus face the added load of economic bias much more than white men and women \citep{BlackIncome}. The top ten median household income in US is \$71,897--\$110,040 while the top ten median African American household incomes range from \$48161--\$68246. The bottom ten median household income in US  is \$48804--\$54546, while that of African Americans is much lower \$27412--\$32696 \citep{BlackEco1}.
Additionally, the percentage of black households with 3-4 person families and 5 or more person families is larger than the percentage of white households, which makes the income per person lower. Similarly, in the US, the percentage of Black and Hispanic single mothers with large families is higher than that of white single mothers \citep{BlackFam}.

 A STEM woman is not a minority in a women’s college because all students are women.  Similarly, in an HBCU an African-American student is not a minority and does not feel the presence of the erroneous and ridiculous negative stereotypes that the larger society has placed in order to keep the status quo. These are parallel experiences. Alversia Wade expresses her sentiments on the how she can be what she wants to be in an HBCU in ``Tell them we are rising". She feels that HBCUs provide a space without boxed in ideas, where you have the freedom to explore who one is, a safe space to be and discover oneself. Overall, HBCUs and Women's Colleges have had a very positive effect in increasing diversity in the STEM workforce. HBCUs are the institution of origin among almost 30\% of black graduates of science and engineering doctorate programs.
%https://www.uncf.org/the-latest/the-impact-of-hbcus-on-diversity-in-stem-fields
Statistics how that HBCUs graduated 46 \% of black women who earned degrees in STEM disciplines between 1995 and 2004 \citep{HigherEdBlack}. Eight HBCUs were among the top 20 institutions to award the most Science & Engineering bachelor’s degrees to black graduates from 2008-2012. Black scientists who have broken scientific and political barriers in the US include Dr. Joycelyn Elders, who was Surgeon General, Dr. Alexa Canady, neurosurgeon, Dr. Mae Jamison, astronaut and Dr. Gladys West, a mathematician who was part of the team of scientist who worked on mathematical models for the Earth used for the Global Positioning System. 

%Racial minority issues have a different dynamic than gender issues. Underserved communities are underserved right from the beginning while women in STEM seem to have this happen post middle school. Nevertheless there are parallels: role of women's colleges and HBCUs, history of exclusion, need for intervention programs etc. 
%Gender parity is less of of problem there. In HBCUs the ratio of black women: black men is reported to 19.2:23 while the same ratio is about 1:2 in all institutions and 1:3 in for-profit institutions.
%https://www.insidehighered.com/news/2015/10/28/survey-finds-big-differences-between-black-hbcu-graduates-those-who-attended-other

Unfortunately, even within women-only colleges and HBCUs fields that are associated with being ``smart" have less representation than fields associated with ``perseverance". In Mount Holyoke College, the Physics department (5 faculty in physics, and 2 in astronomy) is clearly smaller than biology (13 full time faculty and 2 visiting lecturers). Wellesley  also has far more students in biological sciences than in physics \citep{Wes}. Some HBCUs have comparable biology and physics departments, but most do not, e.g., Morehouse College is a male liberal arts college with 12 biology faculty and 10 physics and engineering faculty, while Xavier University has 12 biology faculty and 5 physics full time faculty. Another issue is that, in many colleges, a significant fraction of the teaching is done by adjunct professors who are underpaid and have fewer rights that full professors. %https://www.wellesley.edu/oir/factbook17/fall-enrollment-by-declared-major
%Note that there are minorities within Women's Colleges, e.g., Mount Holyoke has 12-14\% international students and 35-45\% domestic minorities. An additional challenge is that colleges like Mount Holyoke and some of the top HBCUs are very expensive.

\section{Optimal Performance: Immigrants, gender equality, and superdiversity}
\label{Sec5}
A gender equal society encourages innovation. French (11.7\%) and Russian (15.7\%) female inventors are a long way ahead of Japanese (3.7\%), Korean (4.4\%) and German (5.5\%) female inventors. British (7.3\%) and American (8.7\%) female inventors are relatively close to the worldwide average of 7.2\% \citep{nager2016demographics}.

Immigrants comprise a large and vital component of U.S. innovation: 35.5\% of U.S. innovators were born outside the United States. Another 10\% of innovators have at least one parent born abroad. Over 17\% of innovators are not U.S. citizens, and are nonetheless making invaluable contributions to U.S. innovation \citep{nager2016demographics}. {\it Immigrants born in Europe or Asia are over five times more likely to have created an innovation in America than the average native-born U.S. citizen.} Immigrant innovators are also better educated on average than native-born innovators, with over two-thirds holding doctorates in STEM subjects \citep{nager2016demographics}. In part, this may be because there is often a selection process for foreign-born innovators where the ones with the most talent (and perhaps most motivation) choose to come to America because of the significant opportunities this country promises for innovators. 

Women represent only 12\% of U.S. innovators. This constitutes a smaller percentage than the female share of undergraduate degree recipients in STEM fields, STEM Ph.D. students, and working scientists and engineers. The average male born in the United States is nine times more likely to contribute to an innovation than the average female. The United States is therefore missing an enormous potential source of innovation by not creating a gender equal society. Even at this low level, however, the United States outperforms Europe. U.S.-born minorities (including Asians, African Americans, Hispanics, Native Americans, and other ethnicities) make up just 8\% of U.S.-born innovators. However, these groups total 32\% of the total U.S.-born population. Despite comprising 13 \% of the native-born population of the United States, African Americans comprise just half a percent of U.S.-born innovators \citep{nager2016demographics}. Here, too, is an untapped resource of great promise. 

Since many women in science are immigrants with limited visa statuses and human rights, they can more easily become trapped in abusive relationships with their co-workers and academic advisors (or principal investigators). Most subfields have small communities where it can be very difficult to impossible to escape abusers without quitting the field. If they cannot graduate and find a job, which is already challenging with the full support of one's supervisor, an immigrant has to leave the country in shame. This adds pressure and discourages students from seeking help. When they do seek help, they find that complaints (e.g., Title IX) are worded in ways that ensure anonymity for the institution and the abuser while providing little help or protection for the persons being abused.

 Studies find progress is optimal when a super-diversity is maintained, where the teams are not dominated by a single gender or by one or two national identities \citep{page2007making}. They suggest we limit our achievements by limiting diversity. Europe has a strong culture of promoting its own nationals that is partly justified through language barriers. On the other hand, the US, as a new country, included people with a wide range of national identities and  upbringings. Until recently, its openness attracted a diverse scientific community. As a consequence, it still holds the most productive scientific community that exists today. However, the low number of women and minorities, emphasize that the community is not as diverse as it could be, and thus does not maximize creativity. The need for superdiversity has yet to be embraced by scientific communities across the world. %that included the best and the brightest scientists from around the world.



%At first it appears that women originating from countries that focus less on gender equality are more likely to join STEM than their male counterparts. However, data shows that while Asian women are much more likely to join the workforce than white women,
%5.756 x 10^6 in STEM in 2010
% US population 309.6 x 10^6 in 2010
% 0.32 x 309.6 x 10^6 = 99 x 10^6 white men
% 2.94 x 10^6 white men in STEM, about 3% of the general population
% It appears that people who originate from countries who focus less on gender equality like Asia and Eastern Europe are more likely to join the STEM workforce.
 
%In particular, the attitude taken by leaders of institutes and universities is that there is no need to struggle to increase diversity when the institution is already attracting talent and money. This is often stated openly. Unfortunately, studies show that the gender gap in STEM is not diminishing.

\section{Conclusions}
The self-doubt women and minorities experience appears to originate in the attitude towards women and minorities, and towards science \citep{else2013math,organisation2015abc}. Unlike modeling, sport and acting, science is still presented in abstract, uninteresting ways that have not changed for centuries.  Traditionally, scientists are portrayed as dysfunctional men with crazy hair who manage to make use of random numbers, strange objects and outdated materials \citep{fralick2009middle}, while the life of actors and models is seen as glamorous and desirable; yet it has its own problems and stories of abuse.   This portrayal together with other choices made for children starting in early childhood discourages girls and most minorities from all countries and cultures from understanding science and engineering.

The exposure of children to science is larger in Asia and Eastern Europe where the primary and secondary education focus on this end and families pressure both men and women to succeed in these fields. The result is that almost 50\% of women obtain science degrees, but the number of women who attain engineering and technology degrees is under 30\% \citep{India2017,Eurostat,SheNumbers}. The latter number appears to be lower due to hidden and overt biases coupled with the expectation that men should be the providers of the family.

We advocate gender neutral toys, and the exposure of all children to science from the very beginning.  Science can be part of bed-time stories, and part of the curriculum starting in kindergarten and pre-school. Girls and boys should be taught to use their natural curiosity and some of the knowledge accumulated from other people to understand how the world works. The final goal, however, should be to have as many people as possible choosing what they would like to pursue and to be productive and innovative in their chosen fields. 

We advocate that industry, academia and families work together to expose children and students to science and provide mentoring that evolves with age. The attitude of colleagues is less overt against women and minorities than it was in the past, but it is still problematic and the numbers are changing slowly enough that gender parity will not be reached in fields like physics and engineering without intervention programs. It was shown that through intervention programs parity can be achieved even at schools that are highly technical like the \cite{MITpress}. Gender parity in the undergraduate population in mechanical engineering was achieved through deliberate structural changes that included 
\begin{itemize}
\item reaching out to find and invite talented women and then hiring them as faculty instead of waiting for them to find the courage to apply to MIT, 
\item promoting intervention programs like Women in Technology that invite talented high school women to campus for a summer to take part in laboratories and courses, and 
\item modernizing their curriculum in both content and pedagogy, which increased attendance in and was beneficial to both genders.
\end{itemize}


In order to build a gender equal society that taps into the potential offered not just by white men, but also by women, people of color, and other minorities, people should be celebrated and appreciated for their difference in thinking. To hire minorities, active recruiting of talented individuals and encouragement to apply plays a crucial role \citep{2018Report}. To reduce the toxicity of the environment typical traits that are today recognized as being feminine such as empathy and compassion need to be recognized as core values in both the corporate culture and the academia, and not be presented as weaknesses. The ability to nurture a culture of compassion, connect on a personal level and coach people needs to be rewarded to enable and retain not only women, but the best and most talented professionals. Work-place approaches that have proved effective in retaining employees and in reducing gender bias promote
\begin{itemize}
\item gender bias training for all employees,
\item regular round tables with senior women engineers who provide role models for younger women, and 
\item special assignments where women can continue to learn new skills while working with mentors outside of their sub-group. 
\end{itemize}

To improve the environment, intervention programs have to go hand in hand with enforcing policies that allow for sick leave and maternity/paternity leave for employees in industry and academia.  Not providing such leaves is an infringement on human rights that should no longer be the norm. It has been shown that providing employees with maternity leave  can be life-saving \citep{ceron1996risk,del2012intrafamily,eidelman2012breastfeeding} and is associated with long-term health benefits and higher achievements \citep{berger2005maternity,carneiro2015flying}. In the academia, graduate students and postdoctoral students are temporary employees. They are expected to have worked abroad to qualify for faculty positions. Temporary employees do not have the same rights as staff, and if they hold a visa, they are not eligible for maternity and sick leave available to citizens even in countries where such leaves are the norm. Changes have to occur to support not only faculty in top universities to have families, but also students and postdoctoral scholars. All grants have to include provisions for basic human rights for graduate students and postdoctoral scholars so that universities do not rely on workarounds that allow many women to fall through the cracks. %When a degree is obtained, whether it is a BA/BS, MA/MS or PhD, it should not be seen as an end at which point all interest in the student is lost, but as a new beginning where investment is made towards a smooth transition to the next level.

%right now STEM technical and engineering knowledge is mostly with males. 
In the end intervention programs must include the education of males as well \citep{stoet2019simplified}. One reason all girls programs are going to only be part and not the whole solution is that at the moment the STEM technical knowledge and skills are mostly acquired by men. Women must be able to partake of the knowledge gained so far through interactions with both genders. However, women are often made to feel they are not capable of gaining this knowledge and there is hostility and ridicule when they try to gain it. This isolation can be removed in all female settings, however, the full benefit of access to the knowledge community will not be there. Hence changes in attitudes of both men and women are needed so that women feel comfortable as equal learning and working partners in STEM.

Since more than half of the women who obtain a STEM degree switch to other fields mid-career, particular support is needed in the period post-graduate school \citep{hewlett2008athena,huyer2015gender,2018Report}.  While attaining the PhD, BS or MS may seem as the end goal, it is just the beginning of a professional STEM career. It should not be seen as an end at which point all interest in the student is lost, but as a new beginning where investment is made towards a smooth transition to the next level. To lose less talent, steps can be taken to
\begin{itemize}
\item create a network that connects students to the next level, \item allow for maternity/paternity leave and return to work, \item build support networks for students and employees,  
\item actively train students and employees with mentors outside their subgroup.
\end{itemize}
Ultimately, an increase in diversity where teams are not dominated by a single gender and by one or two national identities has been shown to lead to increased productivity, and to decrease the likelihood of abuse. 

%Maternity leave is a basic human right that should not be violated so blatantly because its violation causes long-term health problems for both mothers and children. 
\section*{Acknowledgements}
We thank Kacey Bray Acquilano for support and for providing us with Cornell physics data. We also acknowledge Wellesley and Mount Holyoke staff. We are very grateful to Prof. Andrew P. Lundgren, Dr. Mihai Bondarescu, Rowena Hazell, Dr. Teresia Mansson for taking the time to read our manuscript at various stages and provide suggestions and advice. Lastly, we thank Dr. Claudio Bogazzi for his encouragement and patience as we were writing this paper.

%\section*{Supplemental Data}
% \href{http://home.frontiersin.org/about/author-guidelines#SupplementaryMaterial}{Supplementary Material} should be uploaded separately on submission, if there are Supplementary Figures, please include the caption in the same file as the figure. LaTeX Supplementary Material templates can be found in the Frontiers LaTeX folder.

%\section*{Data Availability Statement}
%The datasets [GENERATED/ANALYZED] for this study can be found in the [NAME OF REPOSITORY] [LINK].
% Please see the availability of data guidelines for more information, at https://www.frontiersin.org/about/author-guidelines#AvailabilityofData

\bibliographystyle{frontiersinSCNS_ENG_HUMS} % for Science, Engineering and Humanities and Social Sciences articles, for Humanities and Social Sciences articles please include page numbers in the in-text citations
%\bibliographystyle{frontiersinHLTH&FPHY} % for Health, Physics and Mathematics articles
\bibliography{WomenInScience}

%%% Make sure to upload the bib file along with the tex file and PDF
%%% Please see the test.bib file for some examples of references

%\section*{Figure captions}

%%% Please be aware that for original research articles we only permit a combined number of 15 figures and tables, one figure with multiple subfigures will count as only one figure.
%%% Use this if adding the figures directly in the mansucript, if so, please remember to also upload the files when submitting your article
%%% There is no need for adding the file termination, as long as you indicate where the file is saved. In the examples below the files (logo1.eps and logos.eps) are in the Frontiers LaTeX folder
%%% If using *.tif files convert them to .jpg or .png
%%%  NB logo1.eps is required in the path in order to correctly compile front page header %%%

%\begin{figure}[h!]
%\begin{center}
%\includegraphics[width=10cm]{logo1}% This is a *.eps file
%\end{center}
%\caption{ Enter the caption for your figure here.  Repeat as  necessary for each of your figures}\label{fig:1}
%\end{figure}


%\begin{figure}[h!]
%\begin{center}
%\includegraphics[width=15cm]{logos}
%\end{center}
%\caption{This is a figure with sub figures, \textbf{(A)} is one logo, \textbf{(B)} is a different logo.}\label{fig:2}
%\end{figure}

%%% If you are submitting a figure with subfigures please combine these into one image file with part labels integrated.
%%% If you don't add the figures in the LaTeX files, please upload them when submitting the article.
%%% Frontiers will add the figures at the end of the provisional pdf automatically
%%% The use of LaTeX coding to draw Diagrams/Figures/Structures should be avoided. They should be external callouts including graphics.

\end{document}
